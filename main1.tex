\documentclass{article}

% Language setting
% Replace `english' with e.g. `spanish' to change the document language
\usepackage[english]{babel}

% Set page size and margins
% Replace `letterpaper' with `a4paper' for UK/EU standard size
\usepackage[letterpaper,top=2cm,bottom=2cm,left=3cm,right=3cm,marginparwidth=1.75cm]{geometry}

% Useful packages
\usepackage{amsmath}
\usepackage{graphicx}
\usepackage[colorlinks=true, allcolors=blue]{hyperref}
\usepackage[final]{pdfpages}
\usepackage{fancyhdr}
\usepackage{setspace}
\usepackage[shortlabels]{enumitem}
\usepackage{lastpage}

\onehalfspacing
\setlength{\parskip}{0.5\baselineskip}%
\setlength{\parindent}{0pt}%

\renewcommand{\headrulewidth}{.0mm} % header line width

\pagestyle{fancy}
\fancyhf{}
\fancyhfoffset[L]{1cm} % left extra length
\fancyhfoffset[R]{1cm} % right extra length
% \rhead{\today}
\lhead{\small Permanent Residence Petition for M.E. Wenbin Hu}
\rfoot{Page \thepage ~of \pageref*{LastPage}}

% \title{Your Paper}
% \author{You}

\begin{document}
% \maketitle

% \begin{abstract}
% Your abstract.
% \end{abstract}
\vspace*{\fill}
\begin{center}

{\bf 
Immigrant Petition for Alien Worker\\
for the Alien with Exceptional Ability in Science (EB2-NIW)
}

\end{center}
\vspace*{\fill}

\begin{center}
TABLE OF CONTENTS
\end{center}
\begin{itemize}
    \item [p. \pageref*{G-1145}] Form G-1145 e-Notification of Application/Petition Acceptance. 
    \item [p. \pageref*{I-140}] Form I-140, Immigrant Petition for Alien Worker with the \$700 filing fee.
    \item [p. \pageref*{I-907}] Form I-907, Request for Premium Processing Service with the \$2500 filing fee.
    \item [p. \pageref*{docs}] Photocopies of the passports, F-1 visas, Form I-20, Form I-94, EAD cards.
    \item [p. \pageref*{IE}] Initial Evidence in Support of the I-140 Immigrant Petition.
    \item [p. \pageref*{plans}] Statement from M.E. Wenbin Hu detailing plans on how he intends to continue work in the United States.
    \item [p. \pageref*{exhib}] Exhibits 1–30.
\end{itemize}

\clearpage
\label{G-1145}
\includepdf[pages=-,pagecommand={},width=1.3\textwidth]{./forms n docs/g-1145.pdf}

\label{I-140}
\includepdf[pages=-,pagecommand={},width=1.3\textwidth]{forms n docs/i-140.pdf}

\label{I-907}
\includepdf[pages=-,pagecommand={},width=1.3\textwidth]{forms n docs/i-907.pdf}

\vspace*{\fill}
\begin{center}

{\LARGE \bf
Title Page of the Current Passport
}
\label{docs}
\end{center}
\vspace*{\fill}

 
% \includepdf[pages=-,pagecommand={},width=1.3\textwidth]{forms n docs/Int Pass 27 Front Page.pdf}

\vspace*{\fill}
\begin{center}

{\LARGE \bf
The Current Student Visa (F-1)\\
in an expired passport
}

\end{center}
\vspace*{\fill}


% \includepdf[pages=-,pagecommand={},width=1.3\textwidth]{forms n docs/F1 2023.pdf}

\vspace*{\fill}
\begin{center}

{\LARGE \bf
Title Page of the Expired Passport\\
and expired US visas
}

\end{center}
\vspace*{\fill}


% \includepdf[pages=-,pagecommand={},width=1.3\textwidth]{forms n docs/Int Pass 23.pdf}

\vspace*{\fill}
\begin{center}

{\LARGE \bf
Current form I-20
}

\end{center}
\vspace*{\fill}


% \includepdf[pages=-,pagecommand={},width=1.3\textwidth]{forms n docs/SignedSTEM_I-20.pdf}

\vspace*{\fill}
\begin{center}

{\LARGE \bf
Current form I-94
}

\end{center}
\vspace*{\fill}


% \includepdf[pages=-,pagecommand={},width=1.3\textwidth]{forms n docs/I94.pdf}

\vspace*{\fill}
\begin{center}

{\LARGE \bf
Current Employment Authorization Document
}

\end{center}
\vspace*{\fill}


% \includepdf[pages=-,pagecommand={},width=1.3\textwidth]{forms n docs/EAD 2023.pdf}

\vspace*{\fill}
\begin{center}

{\LARGE \bf
Expired Employment Authorization Document
}

\end{center}
\vspace*{\fill}

.

% \includepdf[pages=-,pagecommand={},width=1.3\textwidth]{forms n docs/EAD 2022.pdf}



\begin{flushright}
M.E. Wenbin Hu\\
999 99th st,\\
Default City, FL, 99999\\
Tel. (999) 999-9999
\end{flushright}

December 31, 2024

\label{IE}
% Premium Processing
% USCIS Nebraska Service Center
% P.O. Box 87103
% Lincoln, NE 68501-7103

USCIS Nebraska Service Center\\
850 S. Street, Lincoln, NE 68508

\underline{\bf Initial Evidence in Support of the I-140 Immigrant Petition}

\begin{tabular}{ll}
{\bf Petitioner and Beneficiary:} & Wenbin Hu \\
{\bf Classification Sought:} & Employment-Based Immigration, Second Preference, \\
& Exceptional Ability in Science \\
& with a “national interest waiver” of the job offer (EB2-NIW).\\
& Sec. 203(b)(2)(B) INA [8 U.S.C. 1153].
\end{tabular}
\vspace{2\baselineskip}

To Whom It May Concern:

This initial evidence is the attachment to M.E. Wenbin Hu’s I-140 Immigrant Petition for Alien Worker. This evidence shows that M.E. Hu is an alien of exceptional ability in the
engineering, specifically in the application of advanced machine learning methodologies to epileptic seizure prediction and classification, and prospectively will substantially benefit the national economy, educational interests, and welfare of the United States ({\it Please refer to Sections 1, 2 and 3}).

M.E. Hu provides evidence that he satisfies three (A, D, F) of six criteria listed in 8 CFR, Section 204.5(k)(3)(ii), namely:
\begin{enumerate}
    \item M.E. Hu has an advanced degree in Control Engineering from a university in China. ({\it Please refer to Sections 1.1 and 3.2})
    \item M.E. Hu has commanded remuneration for his services, which demonstrates exceptional ability; ({\it Please refer to Section 1.2})
    \item Evidence of M.E. Hu's recognition for achievements and significant contributions to the field by peers and professional organizations. ({\it Please refer to Section 1.3})
\end{enumerate}
Due to the specifics of the highly-competitive area of M.E. Hu's occupation, M.E. Hu additionally provides evidence that he satisfies the following two (iv, vi) of ten criteria listed in 8 CFR, Section 204.5(h)(3) for determination of the extraordinary abilities. 
\begin{enumerate}
    \item M.E. Hu has participated, both individually and on a panel, as a judge of the work of others in the fields of Control Systems, Electrical Engineering, and Artificial Intelligence. ({\it Please refer to Section 1.4})
    \item M.E. Hu has authored scholarly articles in the fields of Control Systems, Cognitive Systems, and Artificial Intelligence. ({\it Please refer to Section 1.5})
\end{enumerate}

The criteria listed in 8 CFR, Section 204.5(h)(3) are comparable to the criteria listed in 8 CFR, Section 204.5(k)(3)(ii) due to the standards of exceptional ability being lower than the standard for extraordinary ability classification.

M.E. Hu is seeking a national interest waiver of the job offer, as per 8 USC 1153(b)(2)(B)(i) and 8 CFR  204.5(k)(4)(ii). The attached evidence and statement satisfy all three criteria for such a waiver, as described in Matter of Dhanasar, 26 I\&N Dec. 884 (AAO 2016). The supporting documentation and statement are as follows:

\begin{enumerate}
    \item M.E. Hu’s proposed work in Artificial Intelligence and Biomedical Engineering has both substantial merit and national importance ({\it Please refer to Section 2})
    \item M.E. Hu is well-positioned to advance the proposed endeavor due to his expertise. ({\it Please refer to Sections 1 and 3})
    \item On balance, it would be beneficial to the United States to waive the job offer and labor certification requirements for M.E. Hu. ({\it Please refer to Sections 2 and 4, and Statement from M.E. Wenbin Hu detailing plans on how he intends to continue work in the United States})
\end{enumerate}

In the United States, M.E. Hu plans to continue to work in the area of expertise. ({\it Please refer to the Statement from M.E. Wenbin Hu detailing plans on how he intends to continue work in the United States and to Exhibits 8 and 9, his current job offers.})

Pursuant to 8 CFR, Section 204.5(k)(1), M.E Hu may file a petition on Form I-140 for classification under Section 203(b)(2) of the Act as an alien of exceptional ability in the sciences on his own behalf because he is seeking an exemption from the requirement of labor certification in the United States pursuant to Section 203(b)(2)(B) of the Act.

\clearpage

{\bf \underline{Section 1.} M.E. Hu is an alien of exceptional ability in the medical artificial intelligence who will prospectively substantially benefit the national economy, educational interests, and welfare of the United States.}

M.E. Hu's main area of study focuses on addressing a significant and pervasive health challenge: epilepsy and the unpredictable seizures it causes. By leveraging state-of-the-art computational techniques to predict seizures more accurately, the Beneficiary’s research aims to enable proactive patient care, reduce the burden on the U.S. healthcare system, cut costs, improve quality of life for patients, and maintain the United States’ global leadership in medical innovation and biomedical engineering research. His planned endeavor is based on top of his past work and aims to advance these sub-fields even further.

 
{\bf 1.1 M.E. Hu has received degrees, including a B.Sc. degree in Electrical Engineering and Automation and a Master’s degree in Control Engineering  from high-ranking Universities. }

Wenbin Hu obtained his Bachelor of Engineering and Master of Engineering Degree from the Hangzhou Dianzi University. ({\it Exhibit 1, the Applied Mathematics and Physics Diploma of Wenbin Hu, and its translation into English.}) According to the ARWU Ranking 2024, it was near the top 3\%  best university in China (97/3074) and one of the Top 600 universities in the world. Specifically for the control automation professional ranking, the university is ranked top 150({\it Exhibit 2, Academic Ranking of World Universities 2024.}, {\it Exhibit 3, Statistical Bulletin on the Development of National Education in  China}) 

His graduate-level performance was exceptional, achieving a high GPA (4.43/5.0 or 89.9/100) and ranking 4th out of 172 students, culminating in an “Excellent” rating for his Master’s dissertation.({\it Exhibit 3, The transcript of the graduate.}) These academic achievements attest to his mastery of complex theoretical frameworks, mathematical modeling skills, and engineering principles.


{\bf 1.2. M.E. Hu has commanded a high compensation for services, demonstrating exceptional ability. }

During his first year at Huawei Technologies Co., Ltd., Mr. Hu received more than CNYXXX as a software engineer, including the basic salary (CNYXXX), bonus (CNYXXX), but excluding benefits ({\it  Exhibit 4. proof of the company's income}). The annual income after three years of work reaches RMB xxx, and the annual income after five years of work reaches RMB xxx. ({\it Exhibit 5, third year and fifth year of proof of income.}). To illustrate the extent to which mr hu's salary is superior to the top experts in his field, his salary can be compared to the 90 percentile of wages reported by china's bureau of labor statistics in the closest region, and has maintained the top salary in the same industry since graduation. As a result, Mr. Hu's salary is X.XX RMB higher than the salary of the country's top computer and information research scientists (XX RMB, XXX RMB) and X.XX RMB higher than the salary of the best bioengineers (XX RMB, XXX RMB). It is X.XX times of the most outstanding electrical engineer in China (XX RMB, XXX). (Annex 6. Report of the Bureau of Labor Statistics of China, ({\it Exhibit 6. Average Salary for Master's Degree Graduates in 2019, Average Salary for Three Years of Work in 2022, and Average Salary for Five Years of Work in 2024})
 
% Conditioning on the location of M.E. Hu's work Hus not change the calculation significantly. According to the Occupational Employment and Wage Statistics data released by the Employment Development Department of the State of California, the 75\% wage percentiles of the relevant occupations in California do not deviate significantly from the nationwide values when compared to the gap with the compensation of M.E. Hu. The 75\% hourly pay percentiles in California for Electrical Engineers, Operations Research Analysts and Industrial Engineers are \$80.44, \$61.58 and \$64.07 accordingly, which essentially coincides with the nationwide statistics: \$63.89, \$56.25 and \$54.80. ({\it Exhibit 20,  Occupational Employment and Wage Statistics by Employment Development Department of the State of California}) This comparison implies that the statistics for the 90\% wage percentiles are also very similar. 

The illustrated major difference in M.E. Hu's compensation from the job market's standards comes from the competitive advantage that M.E. Hu has in the China job market due to his exceptional abilities in his fields of expertise. 

{\bf 1.3 M.E. Hu is recognized for achievements and significant contributions to the fields of Epileptic Seizure Prediction and Machine Learning by peers and professional organizations. }

{\bf 1.3.1 Other scientists recognize M.E. Hu’s exceptional knowledge of Epileptic Seizure Prediction and Machine Learning and consider M.E. Hu a top expert in these fields.}

M.E. Hu's international recognition is evident from the 6 letters supporting his petition that he received from six distinguished professionals from the China and abroad. ({\it Supporting Letters; Exhibits 2–7.}) 

All authors of supporting letters are recognized experts in the fields of Epileptic Seizure Prediction or Machine Learning. Two of them have been M.E. Hu’s mentors, while the other two have never worked with M.E. Hu directly but know his work from his publications and collaborative projects. 

“What sets Hu apart is his rare blend of technical expertise and applied problem-solving acumen. This unique skill set, which goes beyond standard engineering or data science competencies, is not easily replicated. The United States stands to gain substantially from his immediate and unrestricted involvement in ongoing research endeavors, clinical collaborations, and the development of next-generation medical devices.” ({\it Exhibit 4, a letter from Professor B})

“Through sophisticated CNN architectures and multi-layered deep learning frameworks, he has significantly improved the sensitivity and specificity of seizure prediction models we are developing. This improvement is not a marginal enhancement—it can fundamentally change clinical approaches to epilepsy management. By enabling preemptive intervention and careful resource allocation, these methods align seamlessly with U.S. objectives to enhance preventive care, mitigate chronic disease burdens, and reduce the overall financial strain on our healthcare system.” ({\it Exhibit 5, a letter from Dr. C})

“The capacity to predict seizures before they occur is a transformative goal for neurology. By reducing unpredictability, we can lower emergency admission rates, refine treatment schedules, and empower patients with greater autonomy. The United States has invested extensively in health technologies that promote prevention, cost-effectiveness, and improved patient outcomes. Hu’s contributions exemplify these priorities, moving beyond traditional algorithms to extract previously inaccessible predictive insights from complex EEG signals.” ({\it Exhibit 7, a letter from Professor E.})

“Epilepsy management is an enduring challenge that consumes significant healthcare resources and often results in acute patient distress. By employing CNN-based deep learning methods to identify preictal EEG patterns with unprecedented accuracy, Hu has expanded our toolkit for anticipatory patient care. This improvement is no small feat. It reflects the kind of breakthrough that advances U.S. national interests in reducing healthcare expenditures, improving patient safety, and strengthening our capabilities in precision medicine.” ({\it Exhibit 6, a letter from Dr. D.})

{\bf 1.3.2 M.E. Hu has made original discoveries in Epileptic Seizure Prediction or Machine Learning. }

“[...] Hu has made highly distinctive contributions to this field by integrating sophisticated Convolutional Neural Network (CNN) architectures, including stacked deep learning frameworks, to identify subtle preictal EEG biomarkers. His work, substantiated by peer-reviewed publications, not only surpasses conventional analytical approaches but also establishes new benchmarks for predictive accuracy and reliability.” ({\it Exhibit 2, a letter from Professor A.}) 

“Hu’s work in deploying stacked CNN architectures and refined feature extraction methods has substantially advanced our collective ability to forecast seizures. This step forward aligns seamlessly with national interests, as the U.S. healthcare system continually seeks innovative tools to improve patient care, minimize acute interventions, and contain escalating costs. His research directly supports these aims by turning complex EEG data into actionable clinical intelligence.” ({\it Exhibit 5, a letter from Dr. C.}) 

{\bf 1.3.3 Many scientists highly cite the papers co-authored by M.E. Hu. }

The significance and the impact of M.E. Hu’s work are demonstrated by the fact that his papers have been cited at least 187 times by xx  research groups from the United States and other countries, according to citation reports from the Google Scholar citation database. ({\it Exhibit 15, citation reports for M.E. Hu’s papers.}) This number is constantly growing at rates higher than the impact factors of some of the corresponding journals. It is impressive for a young scientist who published his first paper merely 5 years ago when he was an graduate student. Wenbin Hu’s papers have been cited by Professor H, the Vice Chancellor for Research and Distinguished Professor of Electrical Engineering and Computer Science at the University of H, by Prof. J, a professor of Machine Learning in the Computational and Biological Learning Lab, Department of Engineering, University of J, and Prof. K, a top researcher on Optimization in Machine Learning and AI, the Moorthy Family Professor in the departments of Mathematics, Statistics, and the Allen School of Computer Science and Engineering at the University of K.

“I have not interacted with M.E. Hu directly, but I know his papers very well. Our group used Hu's results and I cited Hus work in my publications multiple times.  [...] Our findings were eventually published in the proceedings of the IEEE Conference on Decision and Control, a top conference on Control Theory. This line of work was built on top of the developments published by M.E. Hu [...]” ({\it Exhibit 6, a letter from Dr. D.})


{\bf 1.4. M.E. Hu has been a judge of the work of others in the fields of Machine Learning and Artificial Intelligence. }

M.E. Wenbin Hu completed more than 4 review assignments in the fields of Operations Research and Artificial Intelligence. He was invited to be a reviewer for manuscripts submitted to the Journal of Machine Learning Research, the Open Journal of Mathematical Optimization, the International Conference on Artificial Intelligence and Statistics, the Conference on Decision and Control, the IEEE International Symposium on Information Theory, and others. ({\it Exhibit 16, review assignments completed by M.E. Wenbin Hu and Exhibit 5, a letter from Dr. C})

“[...] we have requested M.E. Hu’s expertise in the area of optimization under uncertainty to review the paper independently and provide feedback to the authors. The review was completed swiftly. M.E. Hu has made valuable suggestions allowing us to accept the manuscript after a revision, delivering an impact to the applications such as signal recovery with uncertainties in the sensing matrix and identification of parameters of time-invariant discrete-time linear dynamical systems via noisy observations.” ({\it Exhibit 5, a letter from Dr. C}) 

M.E. Hu has selected the best researchers in the field of global optimization to present their work at an annual meeting of the main professional organization in Dr.Hu's field of expertise.

“Despite being in the fourth year of his Ph.D., M.E. Hu demonstrated his leadership by assembling an impressive roster of speakers from prestigious institutions. This level of knowledge and dedication to his field was indicative of his exceptional promise as an outstanding scholar and researcher.” ({\it Exhibit 7, a letter from Professor E}) 

M.E. Hu is trusted to judge the work of the top academic researchers who are applying for tenure-track positions at the University of B.

"We have relied heavily on his expertise in evaluating candidates to grow our department further. In March, 2023 M.E. Hu was invited to join recruiting committee for another tenure track position in Renewable Energy as an industry advisor" ({\it Exhibit 4, a letter from Professor B.}) 

{\bf 1.5 M.E. Hu is widely published in the fields of Numerical Optimization, Power Systems, and Artificial Intelligence. His publications have appeared in top journals in these fields. He presented his work at the national and international conferences.}

M.E. Hu has published 4 peer-reviewed articles, with 8 articles as a first author and 3 articles as a second author. He has submitted one more publication for peer review. ({\it Exhibit 13, first pages of 12 papers co-authored by M.E. Hu.}) 

“M.E. Hu’s thesis results are published in prestigious venues as he made key experimental and intellectual contributions to many results produced in my group. Under my supervision, he has published 2 journal articles and 6 conference papers as a first author and co-authored 2 papers as a second author because of his key intellectual contribution to the work. Both of his journal articles are published in the top journals in the fields of mathematical optimization (Mathematical Programming, top-1 h-index overall) and machine learning (Journal of Machine Learning Research, top-1 h-index among open access journals).” ({\it Exhibit 2, a letter from Professor A.}) 

“The work with us is only a very small part of the M.E. Hu’s record since he has co-authored 10 papers before starting his work at our company. That is an impressive work, which summarizes well M.E. Hu’s commitment to his research. I consider myself very fortunate that he accepted to work on our research project.” ({\it Exhibit 3, a letter from F})

According to the Google Scholar Metrics rankings of the venues in Mathematical Optimization, Artificial Intelligence, and Power Engineering, the journals and conferences that published papers by M.E. Hu are at the top of these fields by H-factor in 2023. ({\it Exhibit 24, top venues rankings.}) For example, Mathematical Programming is the top-1 venue for publications on Mathematical Optimization, IEEE Transactions on Smart Grids is the top-1 among the venues on Power Engineering, and AAAI Conference on Artificial Intelligence is the \#4 venue in the ranking for Artificial Intelligence. The Journal of Machine Learning Research is among the top-3 journals on Artificial Intelligence ranked by H-index according to Scimago Journal and County Ranking, which Hus not include conferences.

“Being an active member of the international scientific community, John has been invited to participate in poster sessions and give oral speeches at numerous conferences in the US (<BLABLA>) and internationally (<BLABLABLA>). The main professional society of scientists and industry specialists in Operations Research (INFORMS) has recognized John’s contributions to the field and invited him to organize and chair a session during their Annual Meeting in 2021.” ({\it Exhibit 2, a letter from Professor A.}) 

{\bf 1.6 M.E. Hu has performed a critical role in projects carried out for organizations of distinguished reputation. His knowledge and contribution to the fields have been invaluable to this job and have greatly impacted his area of study. }

“(Project Detail)John played a crucial role in our collaborative projects on the robustness and resiliency of power systems funded by grants from the Office of Naval Research (N00000-00-0-0000), the US National Science Foundation (0000000), the US Army Research Office (W000NF-00-0-0000) and the Air Force Office of Scientific Research (FA0000- 00-0-0000).” ({\it Exhibit 2, a letter from Professor A.}) 

“(Other Dr's letter)Besides this paper, Wenbin Hu co-authored several other high-profile papers during his work as a Ph.D. student with his adviser Prof. A and colleagues from the University of A and other respected universities. Wenbin Hu is assigned as the first author in many of these papers. This distinction means that he was a key person in the published research and carried out most of the intellectual work.” ({\it Exhibit 6, a letter from Dr. D}) 

M.E. Wenbin Hu performs his duties as a Research Scientist at the AI Research and Development department of Anycompany Inc. ({\it Exhibit 9, Anycompany Inc. job offer letter.}) Anycompany is one of the top Artificial Intelligence technology companies in the United States according to the US Rankings citing Bank of America. ({\it Exhibit 23, USNews article on top AI technology stocks.}) His duties include researching safe and robust numerical optimization techniques for training large-scale AI systems.

“The optimization procedure was distributed over thousands of computational machines that had to work synchronously for many months, making the experiment extremely difficult to conduct from a technological point of view. This is where M.E. Hu applied his expertise, becoming responsible for the resiliency and robustness of the computational load to the potential issues coming from the unreliability of the underlying hardware and the limitations of the existing optimization algorithms used in modern AI training. M.E. Hu was surprisingly fast to onboard with the new team and worked collaboratively with our engineers which led to the successful completion of the planned experiments. [...] Working in our group, John has performed a key function and obtained hands-on experience in training large machine learning models that many believe to be the future of AI and the key to the next industrial revolution.” ({\it Exhibit 3, a letter from F}) 


\clearpage


{\bf \underline{Section 2.} M.E. Hu’s proposed employment has both substantial merit and national importance for the United States.}
M.E. Wenbin Hu intends to work in the fields of Epileptic Seizure Prediction and Machine Learning. The following descriptions of federal and state projects provide a summary of the importance and the urgency of the research in  Seizure Prediction:

{\bf 2.1 Epilepsy in the United States: A Pressing Public Health Challenge. }

Epilepsy affects approximately 3.4 million Americans, including nearly 3 million adults and 470,000 children, according to the Centers for Disease Control and Prevention (CDC). Characterized by recurrent seizures that occur unpredictably, epilepsy can result in injuries, hospitalizations, social stigma, and reduced quality of life. Patients and families often must navigate unpredictable episodes, impacting employment, education, and social integration.

Beyond the human toll, epilepsy imposes a significant economic burden. Healthcare costs related to epilepsy run into billions of dollars annually, including emergency room visits, hospital stays, long-term care costs, medications, and indirect costs related to lost productivity and disability. Interventions that reduce seizure frequency, severity, or unpredictability have enormous potential to alleviate these burdens.


{\bf 2.2 Why Seizure Prediction Matters. }

The unpredictability of seizures is a central challenge. Currently, treatments focus on reducing seizure frequency or severity through medications, surgical interventions, or devices like vagus nerve stimulators. But the unpredictable nature of seizure onset often remains. Patients cannot reliably know when a seizure will occur, which elevates risk, anxiety, and sometimes necessitates continuous care or supervision.

Accurate seizure prediction would be transformative. If patients could receive reliable warnings minutes or even seconds before a seizure, they could move to a safe space, prepare emergency medication, or contact support. Clinicians could personalize treatments based on an individual’s EEG profile, adjusting medication dosages or scheduling therapeutic interventions more effectively. This shift from reactive to proactive management aligns closely with U.S. healthcare goals emphasizing preventive care and patient empowerment.

{\bf 2.3 Why Seizure Prediction Matters. }

EEG is a cornerstone tool in epilepsy diagnosis and monitoring. EEG signals capture electrical activity in the brain, providing a non-invasive way to detect abnormal patterns that may herald a seizure. However, raw EEG data is notoriously complex, noisy, and high-dimensional. Human experts can interpret certain patterns, but subtle preictal states are often too subtle or variable to detect with conventional methods.
Machine learning, and deep learning in particular, excel at finding complex patterns in large datasets. CNNs have revolutionized image classification and speech recognition, and now they are being applied successfully to time-series biomedical data. By training CNNs on EEG recordings of patients who have seizures, these models learn to identify signatures that precede seizures. This leads to better accuracy, fewer false alarms, and earlier warnings.

{\bf 2.4 The Beneficiary’s Two Publications and Their Contribution to National Importance. }

1. First Publication in Journal of Ambient Intelligence and Humanized Computing (2023):
The Beneficiary’s initial study focused on mean amplitude spectrum-based epileptic state classification using CNNs. By extracting frequency-domain features (mean amplitude spectrum) and feeding these into CNN architectures, the research achieved more reliable classifications of EEG states, thereby improving the prediction of approaching seizures. This work laid the groundwork for integrating deep learning into clinical epilepsy care. The choice of mean amplitude spectrum features is not trivial—it reflects a deliberate engineering strategy to isolate features associated with imminent seizures. By sharing these results in a reputable international journal, the Beneficiary helped move the epilepsy research community closer to practical, AI-driven seizure prediction tools.

2. Second Publication in IEEE Access (2022): “Epileptic Signal Classification with Deep EEG Features by Stacked CNNs”:
The Beneficiary’s second paper ventured deeper into architecture innovation. He introduced stacked CNNs to learn deep EEG features, effectively capturing multi-scale temporal and spatial dynamics. While the first paper demonstrated that CNN-based classification is effective, the second paper showcased how architectural complexity and multi-layered feature extraction significantly enhance performance. Stacked CNNs can capture more intricate patterns, improving accuracy, sensitivity, and specificity of seizure detection and prediction. Published in IEEE Access, a respected platform known for rapid and open dissemination of peer-reviewed research, this work underscores the Beneficiary’s ability to pioneer more powerful and generalizable solutions. Such advancements are critical for clinical translation and are directly relevant to U.S. medical institutions seeking to integrate high-performing models into patient care.

Besides, the Beneficiary is still working in the group of Cao's group, and We continue to apply the latest technology to epileptic seizure detection, and cooperate with hospitals to apply algorithms to hospital data prediction. The latest algorithm based on xxx is proposed, which verifies the effect of Beneficiary's achievements on the public. These techniques can be applied to the relevant institutions of xx, thus having economic and practical help to xx, balabala.


Together, these two papers represent a trajectory of increasing sophistication, impact, and clinical relevance. They do not merely replicate known methods; they push the boundaries of what machine learning can do with EEG data, paving the way for widespread adoption in U.S. epilepsy care.

{\bf 2.5 Alignment with U.S. National Health and Research Priorities. }

The United States invests heavily in neurological research, often through NIH grants and other federal funding mechanisms. The NIH’s NINDS (National Institute of Neurological Disorders and Stroke) supports studies that improve diagnostics and treatments for epilepsy. Meanwhile, the NIH BRAIN Initiative encourages developing innovative tools and analytical methods to understand brain activity better and treat neurological conditions.

The Beneficiary’s research direction—a stronger, more precise method of EEG-based seizure prediction—resonates with these federal priorities. By improving prediction, we move closer to precision medicine in neurology, where treatments and interventions are tailored based on data-driven insights. This synergy with national strategies ensures that the Beneficiary’s work is not an isolated academic exercise, but directly relevant to ongoing federal and institutional efforts to advance neurological care.


{\bf 2.6 Economic and Strategic Implications. }

From an economic standpoint, breakthroughs in seizure prediction offer a path toward reducing the overall cost of epilepsy on the U.S. healthcare system. As healthcare models shift toward value-based care, interventions that prevent costly ER visits, hospitalizations, and complications align perfectly with national cost-containment and efficiency goals.

Strategically, the U.S. aims to remain a leader in medical AI and biomedical device innovation. Enhanced seizure prediction technologies can fuel the growth of American startups producing EEG-based wearables, telemedicine platforms, and hospital decision-support systems. These businesses would create jobs, attract global customers, and reinforce the U.S. as a hub of medical technology innovation. Thus, improvements in seizure prediction have a ripple effect extending well beyond direct patient care.

{\bf 2.7 Humanitarian and Ethical Dimensions. }

Improving epilepsy care is also a humanitarian imperative. Epilepsy patients often face stigma and constraints that limit their independence. Some cannot drive, work in certain industries, or fully participate in social life due to fear of unpredictable seizures. A reliable prediction tool can restore autonomy and dignity, enabling patients to manage their lives more freely. This aligns with American values of empowerment and equal opportunity.

In summary, the national importance of the Beneficiary’s research is evident at multiple levels: public health improvements, alignment with federal research priorities, economic benefits, strategic technological leadership, and ethical considerations that enhance patient well-being. These intersecting factors firmly anchor the Beneficiary’s work as having substantial merit and national importance.


\clearpage

{\bf \underline{Section 3.} The Beneficiary is well-positioned to advance the proposed endeavor due to his expertise. }

The second Dhanasar prong necessitates that the Beneficiary be well-positioned to advance his endeavor. This means he must have the skills, experience, credibility, and track record to move his research from theory to impactful application. The Beneficiary amply meets these criteria.


{\bf 3.1 Academic Excellence and Engineering Foundations.}

The Beneficiary’s advanced degree (Master’s in Control Engineering) from Hangzhou Dianzi University and high GPA highlight his capacity to master complex concepts. Control Engineering integrates mathematics, systems analysis, and signal processing—the perfect substrate for learning and implementing EEG data analytics. Ranking 4 out of 172 peers and earning an “Excellent” dissertation rating indicate that his professors and mentors recognized his abilities, analytical rigor, and research potential.

Such technical and theoretical competence is crucial, as EEG-based seizure prediction involves intricate pattern recognition, understanding signal theory, statistical modeling, and advanced algorithmic design. Without a strong theoretical grounding, it would be challenging to navigate the complexity of EEG signals and deep learning frameworks. ({\it Please see Exhibit 1, CV of M.E. Wenbin Hu.}) 

{\bf 3.2 Comprehensive Research Experience in EEG-based Seizure Prediction.}

During his time in Professor Cao’s group, the Beneficiary engaged in full-stack EEG analytics. He did not specialize in just one facet; he learned to handle raw EEG signals, reduce noise, extract critical features, and implement machine learning pipelines. This hands-on immersion provided a holistic understanding of the challenges and solutions in epilepsy research.

He explored different machine learning models: CNN, SVM, KNN, and weighted fusion strategies. Such breadth ensured that he could compare models, tune hyperparameters, and integrate multiple architectures for optimal results. This versatility sets him apart from researchers who know only a single modeling approach. The Beneficiary can tailor solutions to diverse data sets and clinical conditions, a vital skill as no “one-size-fits-all” model exists for all EEG variations.


{\bf 3.3 Peer-Reviewed Publications Demonstrating Ongoing Innovation.}

First Paper (Journal of Ambient Intelligence and Humanized Computing):
Demonstrated the feasibility and improved accuracy of CNN-based seizure prediction using mean amplitude spectrum features. Publication in this reputable journal indicates that experts in the field found the methodology sound, the results credible, and the contribution meaningful.

Second Paper (IEEE Access – “Epileptic Signal Classification with Deep EEG Features by Stacked CNNs”):
Signifies a technical leap forward. By adopting stacked CNN architectures, he advanced from a baseline CNN model to a more complex and capable architecture that can learn deeper EEG representations. IEEE Access is known for thorough peer review and rapid dissemination, indicating that the Beneficiary’s work is cutting-edge, timely, and relevant.

These publications prove he is not only capable of conducting research but also of achieving recognized, peer-reviewed breakthroughs. Being first or among the early adopters of stacked CNN architectures for epileptic signal classification places him at the forefront of the field. ({\it Exhibit 4, a letter from Professor B.})

{\bf 3.4 Patent Application: Moving Toward Applied Innovations.}
His contribution to a Chinese patent application (No. 201810116608.9) related to CNN-based pre-seizure detection methods signals a transition from theory to innovation. Patents underscore novelty, utility, and potential commercial viability. By developing intellectual property, he shows readiness to contribute to medical device manufacturing, software solutions, and other commercial channels that could deliver immediate benefits to U.S. healthcare systems upon his involvement.


{\bf 3.5 Technical Skills: Programming, Software, and Documentation.}
The Beneficiary’s comfort with Python, data analysis libraries, Linux systems, and frameworks like TensorFlow ensures quick adaptation to U.S. labs and R\&D environments. Modern medical research relies heavily on open-source software stacks, version control systems, continuous integration, and reproducible machine learning pipelines. His ability to build TensorFlow models from source code if needed, run large-scale EEG data analyses efficiently, and document results thoroughly is essential for high-level collaborations.


{\bf 3.6 Interdisciplinary Expertise and Collaborative Potential.}
Epilepsy research is inherently multidisciplinary, blending neurology, clinical practice, biomedical engineering, signal processing, and machine learning. The Beneficiary’s profile—combining engineering foundations with deep learning advancements—makes him a valuable addition to U.S. teams that may already include neurologists, clinicians, data scientists, and hardware developers. He can communicate effectively across disciplines, aligning technical solutions with clinical realities.

{\bf 3.7 Access to U.S. Data and Resources.}
Once in the U.S., the Beneficiary could partner with NIH-funded epilepsy research consortia or collaborate with major institutions that have large EEG datasets. His methods, refined on these larger and more diverse data sets, could improve generalizability. This environment would let him iterate quickly, produce new models, and test them in clinical pilot studies. Such a fertile environment amplifies his ability to realize the full potential of his research.

In all these respects, the Beneficiary is undoubtedly well-positioned. He combines academic brilliance, proven research output, practical technical skills, and innovation-oriented thinking. He is not merely a student of the field; he is a contributor advancing the state of the art, as evidenced by his publications and patent application.


\clearpage

{\bf \underline{Section 4.} BENEFIT TO THE UNITED STATES OF WAIVING THE JOB OFFER AND LABOR CERTIFICATION REQUIREMENTS. }

The third Dhanasar prong examines whether it is in the United States’ interest to waive the typical labor certification and job offer requirements. This section explains why the national interest would be best served by allowing the Beneficiary to proceed without these constraints.

{\bf 4.1 The Unique and Emerging Nature of the Work.}

Epileptic seizure prediction using advanced CNNs and deep EEG features is a specialized niche. While the U.S. workforce has many capable data scientists and engineers, relatively few possess the Beneficiary’s exact blend of EEG signal processing know-how, deep learning model design, and proven success in academic research and potential clinical translation. Labor certification requires defining a job in standard occupational categories and testing the market for U.S. workers. This works poorly when the job itself—interdisciplinary EEG-based seizure prediction research—is highly specialized and evolving. Waiting months or years for labor certification would stall urgent research efforts.

{\bf 4.2 Prompt Integration into Research and Healthcare Ecosystems.}

The Beneficiary’s work can have immediate applications in ongoing U.S. research projects. For instance, NIH-funded labs might currently be testing various seizure prediction algorithms. A startup developing wearable EEG monitors might be seeking a cutting-edge algorithm to differentiate itself and serve patients better. A major hospital system might be launching a clinical trial exploring AI-driven seizure management strategies. The Beneficiary, freed from the constraints of labor certification, can quickly join these efforts, maximizing impact and ensuring timely results. Any delay could mean missed opportunities to improve patient care or accelerate innovation.

{\bf 4.3 The High Value of Flexibility in National Interest Endeavors.}

The fundamental premise behind the National Interest Waiver is that certain endeavors surpass the benefits of a strict labor market test. Epilepsy research, improved seizure prediction, and the associated potential for cost savings and quality-of-life improvements meet this criterion. Granting the NIW allows the Beneficiary to choose collaborations and projects that yield the greatest national benefit rather than binding him to a single employer’s position. This flexibility ensures his skills can be allocated where they are most needed.


{\bf 4.4 Enhancing U.S. Global Competitiveness.}

Medical AI and neural engineering are hotly contested frontiers. Countries worldwide are racing to attract top talent. If the Beneficiary is hampered by the labor certification process, he might pursue opportunities elsewhere. By granting the NIW, the U.S. secures a researcher who can help maintain and even extend the nation’s leadership in medical technology. The global healthcare AI market is expanding rapidly, and the U.S. stands to gain from having leading experts developing next-generation solutions domestically.


{\bf 4.5 Reducing Healthcare Costs and Improving Patient Outcomes.}

By enabling better seizure predictions, the Beneficiary’s work can reduce unnecessary healthcare expenses. Fewer emergency admissions mean lower costs for hospitals, insurers, patients, and government programs like Medicare and Medicaid. Over time, the widespread adoption of advanced seizure prediction models could translate into billions saved, aligning perfectly with U.S. goals of cost-effective, high-quality healthcare.


{\bf 4.6 Legislative Intent and the Nature of National Interest Waivers.}

National Interest Waivers are intended for cases precisely like this, where a labor market test adds little value and may hinder significant national benefits. The Beneficiary’s research direction is not a commodity skill set. By approving the NIW, USCIS would fulfill the spirit of the relevant statutes and decisions (Dhanasar and prior precedents), facilitating the admission of individuals whose expertise and contributions are vital to national interests.

{\bf 4.7 No Detriment to U.S. Workers.}

It is important to recognize that granting an NIW Hus not harm U.S. workers. The specialized nature of the Beneficiary’s work means he complements existing teams rather than replacing or undercutting them. His arrival could create more opportunities as his research leads to new projects, funding, and technology developments. Ultimately, enhanced epilepsy care benefits not only patients but also the broader economy and the healthcare industry at large.

In essence, the balancing test strongly favors waiving the labor certification requirement. The U.S. gains immediate and flexible access to a uniquely qualified researcher who can advance a nationally significant area of healthcare research and innovation.

\clearpage

{\bf \underline{Section 5.} Concluding Remarks. }

""
“Clearly, M.E. Hu is an exceptionally talented mathematician and engineer, and he would be a great asset to the United States if he continued working here. If he enters academics as a Engineer, as planned, then he will make important contributions to the medical artificial intelligence for State and the US, training future generations of engineers among the Veterans and the native population. If he stays in the industry, he will create new commercial opportunities based on his advanced knowledge of next-generation AI technology and the safety of large computational systems. In short, M.E. Wenbin Hu is already a leader in main areas of immense importance to our economy and national security, and his leadership position will inevitably increase. Please give favorable consideration to his Green Card application.” ({\it Exhibit 2, a letter from Professor A.}) 

“I endorse the immigration petition by M.E. Hu and ask you to decide favorably on his behalf so that he can continue his important research without delays and distractions.” ({\it Exhibit 4, a letter from Professor B.}) 

"M.E. Hu offers a unique skillset to the American scientific community. He is a creative engineer with an unusually keen attention to detail. I strongly support his application." ({\it Exhibit 3, a letter from F.}) 

“Let me finish this letter with the statement that M.E. Wenbin Hu is a brilliant young investigator in the fields of operations research and artificial intelligence. Granting him permanent residence in the U.S. will allow his work to proceed uninterrupted so he can concentrate on the application of his skills and knowledge to solving major computational problems.” ({\it Exhibit 5, a letter from Dr. C.}) 

“I wholeheartedly endorse M.E. Hu's application for the EB-2 NIW petition for Permanent Residence. His advanced research and demonstrated leadership in his field make him a strong candidate, and I am confident that his ongoing contributions will substantially benefit our nation.” ({\it Exhibit 7, a letter from Professor E.}) 

“In my opinion, M.E. Hu has made very significant discoveries in control and signal processing and helped in the advance of operations research. His outstanding abilities and expertise will be a huge asset to the science of computation in the USA. There is no doubt he will continue to have a major impact on electrical engineering, operations research, machine learning, and artificial intelligence. He will certainly contribute substantially to the well-being of American society and help sustain the USA as the leading light in world science.” ({\it Exhibit 6, a letter from Dr. D.}) 

The Beneficiary’s research into EEG-based epileptic seizure prediction using advanced machine learning techniques, including stacked CNNs, directly addresses a critical national interest. By tackling a major healthcare challenge—epilepsy—and offering solutions that could improve patient independence, reduce healthcare spending, and enhance U.S. leadership in medical AI, his work holds substantial merit and national importance.

He has demonstrated, through exemplary academic performance, hands-on research experience, two peer-reviewed international publications, and a patent application, that he is eminently well-positioned to advance this field. He is not a novice but a researcher who has already achieved innovations recognized by the international scholarly community.

Requiring a job offer and labor certification would slow the pace of beneficial collaborations, limit his potential contributions, and ultimately serve as an unnecessary barrier to harnessing his specialized skill set for the U.S. public good. Granting a National Interest Waiver aligns perfectly with the legislative intent behind the NIW category and ensures that his talents can be immediately and fully leveraged to advance public health and scientific innovation in the United States.

For these reasons, I respectfully request that USCIS approve this petition and grant the National Interest Waiver for the Beneficiary. Should there be any need for additional information or clarification, I will be pleased to provide further documentation.
Thank you for your time and careful consideration.

Respectfully submitted,

\vspace{5\baselineskip}

Wenbin Hu\\
999 99th st,\\
Default City, FL, 99999\\
Tel. (999) 999-9999




\clearpage

{\bf Statement from M.E. Wenbin Hu detailing plans on how he intends to continue work in the United States}

\label{plans}
December 31, 2024

My name is Wenbin Hu. I am writing this letter to provide a detailed overview of my plans on how I intend to continue my research and related endeavors in the United States, should my National Interest Waiver (NIW) petition be approved. As outlined in my I-140 petition, my area of focus is improving the prediction and understanding of epileptic seizures through advanced EEG-based machine learning techniques. By bringing my expertise to the U.S., I aim to drive meaningful progress in epilepsy care, contribute to medical AI innovation, and engage with various American institutions and enterprises dedicated to this field.

Below, I present my strategic approach and anticipated activities, collaborations, and contributions, reflecting how I envision establishing myself as a productive and influential researcher in the United States.


{\bf 1. Academic and Clinical Collaborations }

{\bf University Partnerships: }

I plan to affiliate with leading academic medical centers and universities known for epilepsy and neuroscience research, such as Johns Hopkins University, the Mayo Clinic, and the Cleveland Clinic. Working within their neurology and biomedical engineering departments will provide access to large EEG datasets and ongoing clinical trials. This environment will allow me to validate my algorithms against diverse patient populations, refine predictive accuracy, and rapidly incorporate clinical feedback.

{\bf NIH and NINDS Initiatives: }

I aim to align my work with NIH-funded projects, including initiatives by the National Institute of Neurological Disorders and Stroke (NINDS), and possibly the BRAIN Initiative. By joining consortia focused on improving diagnostic and therapeutic approaches for epilepsy, I can integrate my models into federally supported research, ensuring my contributions directly advance U.S. public health goals.

{\bf 2. Industry and Startup Engagement }

{\bf Medical Device Firms and AI Companies: }

The U.S. has a thriving healthcare technology sector. I will seek collaborations with startups and established med-tech companies developing EEG-based monitoring devices or AI-driven healthcare platforms. Integrating my seizure prediction algorithms into their product pipelines will help accelerate commercialization. This approach ensures that the benefits of my research—improved accuracy, timely warnings, and reduced false alarms—are realized swiftly and made available to patients across the country.

{\bf Technology Transfer and Entrepreneurship: } 

If appropriate, I may pursue technology licensing agreements or partner with university technology transfer offices. These pathways could lead to SBIR grants, private investment, or co-founding a startup focused on delivering reliable seizure prediction tools to U.S. clinics, hospitals, and telemedicine services.

{\bf 3. Funding and Grant Strategies }

{\bf Federal Grants (NIH, NSF): }

I will apply for NIH and NSF grants that support innovative diagnostic tools, AI in healthcare, and neurological research. Successful grant proposals will help me scale up my work, fund additional data analyses, and perform long-term clinical validations.

{\bf Foundation and Non-Profit Support: }

I will also seek grants from organizations like the Epilepsy Foundation, which often provide seed funding for innovative approaches. Such awards can support pilot studies and generate preliminary data critical for securing larger federal grants.

{\bf 4. Clinical Implementation and Validation }

{\bf Pilot Studies in U.S. Hospitals: }

My immediate goal is to test my EEG-based models in real clinical environments. Partnering with epilepsy monitoring units will allow me to measure algorithmic performance on patients undergoing routine evaluation, assessing lead times, false positives, and ease of integration into clinical workflows.

{\bf Regulatory Pathways and FDA Compliance: }

I intend to follow FDA guidelines for AI-based medical tools, ensuring that my algorithms meet safety, transparency, and efficacy standards. Early engagement with FDA frameworks will streamline future device approvals, facilitating quicker adoption of prediction tools in everyday patient care.

{\bf 5. Education, Training, and Dissemination }

{\bf Mentorship and Teaching: }

By collaborating with U.S. universities, I can mentor graduate students and research assistants, helping develop a skilled talent pool versed in medical AI and EEG analysis. This capacity-building aligns with U.S. interests in sustaining a robust research community.

{\bf Conferences and Workshops: }

I will present my work at national conferences on biomedical engineering, neuroscience, and AI in healthcare. Sharing methodologies, open-source code, and research findings will foster community engagement, stimulate feedback, and encourage the formation of new collaborations.

{\bf 6. Long-Term Vision and Broader Impact } 

{\bf Extending to Other Neurological Conditions: }

While epilepsy prediction is my initial focus, the underlying deep learning methods I use can be adapted to detect early markers of other neurological disorders. Over time, I plan to expand my scope to conditions like Parkinson’s disease or Alzheimer’s, contributing to a broader range of U.S. health priorities.

{\bf Policy and Standards Contribution: }

As my work matures, I may engage with patient advocacy groups, healthcare policy makers, and regulatory bodies. By advising on best practices and ethics in AI-driven neurology, I can help shape national standards, ensuring that these emerging technologies serve patient interests and align with U.S. healthcare policies.

My strategy for continuing work in the United States is structured, impact-driven, and aligned with national priorities. By establishing partnerships with leading research institutions, contributing to federally funded programs, collaborating with industry, securing diverse funding, achieving clinical validation, and participating in educational and policy discussions, I will ensure that my expertise in EEG-based seizure prediction delivers tangible benefits. This integrated approach will improve patient outcomes, reduce healthcare burdens, enhance American leadership in medical innovation, and advance the nation’s public health objectives.

Thank you for considering this plan. I remain at your disposal for any further details or clarifications.

Respectfully,

[Name of Beneficiary]

\vspace{5\baselineskip}

Wenbin Hu\\
999 99th st,\\
Default City, FL, 99999\\
Tel. (999) 999-9999


\clearpage

{\bf List of Exhibits}
\label{exhib}

\begin{enumerate}[label={Exhibit \arabic*:}]
    \item Curriculum Vitae of M.E. Wenbin Hu 
    \item Letter of Professor A, University of A 
    \item Letter of F, Anycompany Inc. 
    \item Letter of Professor B, University of B at Manoa 
    \item Letter of Dr. C, Lab 
    \item Letter of Dr. D, University of D 
    \item Letter of Professor E, University of E
    \item Job Offer: University of B
    \item Job Offer: Anycompany Inc.
    \item Bachelor diploma of M.E. Hu 
    \item Graduate diploma of M.E. Hu 
    \item Abstract of the Ph.D. dissertation by M.E. Hu 
    \item First pages of 12 papers co-authored by M.E. Hu 
    \item Conference invitations for M.E. Hu 
    \item Citation reports for M.E. Hu’s papers 
    \item Review assignments completed by M.E. Wenbin Hu 
    \item Admission letters from graduate schools to M.E. Hu 
    \item Announcement of the Assistant Professor position offered to M.E. Wenbin Hu 
    \item U.S. Bureau of Labor Statistics report 
    \item Times Higher Education University Ranking 2017 
    \item Times Higher Education University Ranking 2022 
    \item University of B page on the Carnegie Classification of Institutions of Higher Education website 
    \item USNews article on top AI technology stocks 
    \item Top venues in Artificial Intelligence, Mathematical Optimization and Power Engineering 
    \item Description of the Grid Modernization and the Smart Grid project by the Department of Energy 
    \item The White House Greenhouse Gas Pollution Reduction Target 
    \item The description of Clean Energy Initiative by the State Energy Office 
    \item The page devoted to Artificial Intelligence (AI) by the U. S. Department of State
    \item New Actions to Promote Responsible AI Innovation that Protects Americans’ Rights and Safety 
    \item W-2 form M.E. Hu received from Anycompany Inc. for the calendar year 2022
\end{enumerate}

\clearpage

\vspace*{\fill}
\begin{center}

{\LARGE \bf
Exhibit 1
}

\vspace{10\baselineskip}

{\large Curriculum Vitae of M.E. Wenbin Hu}

\end{center}
\vspace*{\fill}

% \includepdf[pages=-,pagecommand={},width=1.3\textwidth]{Exhibits/CV.pdf}

\vspace*{\fill}
\begin{center}

{\LARGE \bf
Exhibit 2
}

\vspace{10\baselineskip}

{\large Letter of Professor A, University of A}

\end{center}
\vspace*{\fill}
f
% \includepdf[pages=-,pagecommand={},width=1.3\textwidth]{Exhibits/recomendations/A.pdf}

\vspace*{\fill}
\begin{center}

{\LARGE \bf
Exhibit 3
}

\vspace{10\baselineskip}

{\large Letter of F, Anycompany Inc.}

\end{center}
\vspace*{\fill}

% \includepdf[pages=-,pagecommand={},width=1.3\textwidth]{Exhibits/recomendations/F.pdf}


\vspace*{\fill}
\begin{center}

{\LARGE \bf
Exhibit 4
}

\vspace{10\baselineskip}

{\large Letter of Professor B, University of B}

\end{center}
\vspace*{\fill}


% \includepdf[pages=-,pagecommand={},width=1.3\textwidth]{Exhibits/recomendations/B.pdf}



\vspace*{\fill}
\begin{center}

{\LARGE \bf
Exhibit 5
}

\vspace{10\baselineskip}

{\large  Letter of Dr. C, Lab}

\end{center}
\vspace*{\fill}


% \includepdf[pages=-,pagecommand={},width=1.2\textwidth]{Exhibits/recomendations/C.pdf}




\vspace*{\fill}
\begin{center}

{\LARGE \bf
Exhibit 6
}

\vspace{10\baselineskip}

{\large Letter of Dr. D, University of D}

\end{center}
\vspace*{\fill}

% \includepdf[pages=-,pagecommand={},width=1.3\textwidth]{Exhibits/recomendations/D.pdf}




\vspace*{\fill}
\begin{center}

{\LARGE \bf
Exhibit 7
}

\vspace{10\baselineskip}

{\large Letter of Professor E, University of E}

\end{center}
\vspace*{\fill}

% \includepdf[pages=-,pagecommand={},width=1.3\textwidth]{Exhibits/recomendations/E.pdf}




\vspace*{\fill}
\begin{center}

{\LARGE \bf
Exhibit 8
}

\vspace{10\baselineskip}

{\large Job Offer: University of Whatever}

\end{center}
\vspace*{\fill}


% \includepdf[pages=-,pagecommand={},width=1.3\textwidth]{Exhibits/offer1.pdf}


\vspace*{\fill}
\begin{center}

{\LARGE \bf
Exhibit 9
}

\vspace{10\baselineskip}

{\large Job Offer: Anycompany Inc.}

\end{center}
\vspace*{\fill}

% \includepdf[pages=-,pagecommand={},width=1.3\textwidth]{Exhibits/offer.pdf}




\vspace*{\fill}
\begin{center}

{\LARGE \bf
Exhibit 10
}

\vspace{10\baselineskip}

{\large Bachelor diploma of M.E. Hu}

\end{center}
\vspace*{\fill}

% \includepdf[pages=-,pagecommand={},width=1.3\textwidth]{Exhibits/BSc diploma/BSc diploma english.pdf}

% \includepdf[pages=-,pagecommand={},width=1.3\textwidth]{Exhibits/BSc diploma/BSc diploma original.pdf}


\vspace*{\fill}
\begin{center}

{\LARGE \bf
Exhibit 11
}

\vspace{10\baselineskip}

{\large Graduate degree diploma of M.E. Hu}

\end{center}
\vspace*{\fill}

% \includepdf[pages=-,pagecommand={},width=1.3\textwidth]{Exhibits/PhD diploma/PhDdiploma.pdf}

% \includepdf[pages=-,pagecommand={},width=1.3\textwidth]{Exhibits/PhD diploma/master diploma.pdf}

% \includepdf[pages=-,pagecommand={},width=1.3\textwidth]{Exhibits/PhD diploma/FINAL TRANSCRIPT.pdf}


\vspace*{\fill}
\begin{center}

{\LARGE \bf
Exhibit 12
}

\vspace{10\baselineskip}

{\large Abstract of the Ph.D. dissertation by M.E. Hu}

\end{center}
\vspace*{\fill}


% \includepdf[pages=-,pagecommand={},width=1.3\textwidth]{Exhibits/Thesis abstract.pdf}



\vspace*{\fill}
\begin{center}

{\LARGE \bf
Exhibit 13
}

\vspace{10\baselineskip}

{\large  First pages of 12 papers co-authored by M.E. Hu}

\end{center}
\vspace*{\fill}


% \includepdf[pages=-,pagecommand={},width=1.3\textwidth]{Exhibits/my papers/1.pdf}

% \includepdf[pages=-,pagecommand={},width=1.3\textwidth]{Exhibits/my papers/2.pdf}

% \includepdf[pages=-,pagecommand={},width=1.3\textwidth]{Exhibits/my papers/Thesis.pdf}

% \includepdf[pages=-,pagecommand={},width=1.3\textwidth]{Exhibits/my papers/3.pdf}

% \includepdf[pages=-,pagecommand={},width=1.3\textwidth]{Exhibits/my papers/4.pdf}

% \includepdf[pages=-,pagecommand={},width=1.3\textwidth]{Exhibits/my papers/5.pdf}

% \includepdf[pages=-,pagecommand={},width=1.3\textwidth]{Exhibits/my papers/6.pdf}

% \includepdf[pages=-,pagecommand={},width=1.3\textwidth]{Exhibits/my papers/7.pdf}

% \includepdf[pages=-,pagecommand={},width=1.3\textwidth]{Exhibits/my papers/8.pdf}

% \includepdf[pages=-,pagecommand={},width=1.3\textwidth]{Exhibits/my papers/9.pdf}

% \includepdf[pages=-,pagecommand={},width=1.3\textwidth]{Exhibits/my papers/10.pdf}

% \includepdf[pages=-,pagecommand={},width=1.3\textwidth]{Exhibits/my papers/11.pdf}


\vspace*{\fill}
\begin{center}

{\LARGE \bf
Exhibit 14
}

\vspace{10\baselineskip}

{\large Conference invitations for M.E. Hu}

\end{center}
\vspace*{\fill}

% \includepdf[pages=-,pagecommand={},width=1.3\textwidth]{Exhibits/presentation notifications/ paper notification.pdf}

% \includepdf[pages=-,pagecommand={},width=1.3\textwidth]{Exhibits/presentation notifications/ presentation notification.pdf}

% \includepdf[pages=-,pagecommand={},width=1.3\textwidth]{Exhibits/presentation notifications/C.pdf}

% \includepdf[pages=-,pagecommand={},width=1.3\textwidth]{Exhibits/presentation notifications/A.pdf}

% \includepdf[pages=-,pagecommand={},width=1.3\textwidth]{Exhibits/presentation notifications/CC.pdf}


\vspace*{\fill}
\begin{center}

{\LARGE \bf
Exhibit 15
}

\vspace{10\baselineskip}

{\large Citation reports for M.E. Hu’s papers}

\end{center}
\vspace*{\fill}

% \includepdf[pages=-,pagecommand={},width=1.3\textwidth]{Exhibits/citation reports/blabla.pdf}


\vspace*{\fill}
\begin{center}

{\LARGE \bf
Exhibit 16
}

\vspace{10\baselineskip}

{\large  Review assignments completed by M.E. Wenbin Hu}

\end{center}
\vspace*{\fill}

% \includepdf[pages=-,pagecommand={},width=1.3\textwidth]{Exhibits/review invites/1 x3.pdf}

% \includepdf[pages=-,pagecommand={},width=1.3\textwidth]{Exhibits/review invites/2.pdf}

% \includepdf[pages=-,pagecommand={},width=1.3\textwidth]{Exhibits/review invites/3.pdf}

% \includepdf[pages=-,pagecommand={},width=1.3\textwidth]{Exhibits/review invites/4.pdf}

% \includepdf[pages=-,pagecommand={},width=1.3\textwidth]{Exhibits/review invites/I5.pdf}

% \includepdf[pages=-,pagecommand={},width=1.3\textwidth]{Exhibits/review invites/I6.pdf}

% \includepdf[pages=-,pagecommand={},width=1.3\textwidth]{Exhibits/review invites/C7.pdf}

% \includepdf[pages=-,pagecommand={},width=1.3\textwidth]{Exhibits/review invites/A8.pdf}

% \includepdf[pages=-,pagecommand={},width=1.3\textwidth]{Exhibits/review invites/A9.pdf}

% \includepdf[pages=-,pagecommand={},width=1.3\textwidth]{Exhibits/review invites/A10.pdf}

% \includepdf[pages=-,pagecommand={},width=1.3\textwidth]{Exhibits/review invites/A11.pdf}

% \includepdf[pages=-,pagecommand={},width=1.3\textwidth]{Exhibits/review invites/A12.pdf}



\vspace*{\fill}
\begin{center}

{\LARGE \bf
Exhibit 17
}

\vspace{10\baselineskip}

{\large Admission letters from graduate schools to M.E. Hu}

\end{center}
\vspace*{\fill}

% \includepdf[pages=-,pagecommand={},width=1.3\textwidth]{Exhibits/admission letters from graduate schools/award notification.pdf}

% \includepdf[pages=-,pagecommand={},width=1.3\textwidth]{Exhibits/admission letters from graduate schools/Fellowship award.pdf}

% \includepdf[pages=-,pagecommand={},width=1.3\textwidth]{Exhibits/admission letters from graduate schools/Letter of Admission.pdf}

% \includepdf[pages=-,pagecommand={},width=1.3\textwidth]{Exhibits/admission letters from graduate schools/email of admission.pdf}

\vspace*{\fill}
\begin{center}

{\LARGE \bf
Exhibit 18
}

\vspace{10\baselineskip}

{\large Announcement of the Assistant Professor position offered to M.E. Wenbin Hu}

\end{center}
\vspace*{\fill}

% \includepdf[pages=-,pagecommand={},width=1.3\textwidth]{Exhibits/search ad.pdf}


\vspace*{\fill}
\begin{center}

{\LARGE \bf
Exhibit 19
}

\vspace{10\baselineskip}

{\large Occupational Employment and Wage Statistics by U.S. Bureau of Labor Statistics}

\end{center}
\vspace*{\fill}

% \includepdf[pages=-,pagecommand={},width=1.3\textwidth]{Exhibits/salary stats/Computer and Information Research Scientists.pdf}

% \includepdf[pages=-,pagecommand={},width=1.3\textwidth]{Exhibits/salary stats/Electrical Engineers.pdf}

% \includepdf[pages=-,pagecommand={},width=1.3\textwidth]{Exhibits/salary stats/Operations Research Analysts.pdf}

% \includepdf[pages=-,pagecommand={},width=1.3\textwidth]{Exhibits/salary stats/Industrial Engineers.pdf}




\vspace*{\fill}
\begin{center}
{\LARGE \bf
Exhibit 20
}

\vspace{10\baselineskip}

{\large Academic Ranking of World Universities 2024}

\end{center}
\vspace*{\fill}

 \includepdf[pages=-,pagecommand={},width=1.3\textwidth]{Exhibits/Academic Ranking of World Universities 2024.pdf}

\vspace*{\fill}
\begin{center}
{\LARGE \bf
Exhibit 21
}

\vspace{10\baselineskip}

{\large Statistical Bulletin on the Development of National Education in China}

\end{center}
\vspace*{\fill}

\includepdf[pages=-,pagecommand={},width=1.3\textwidth]{Exhibits/Statistical Bulletin on the Development of National Education in China.pdf}


\vspace*{\fill}
\begin{center}
{\LARGE \bf
Exhibit 22
}

\vspace{10\baselineskip}

{\large University of Anywhere page on the website of Carnegie Classification\\ of Institutions of Higher Education}

\end{center}
\vspace*{\fill}


% \includepdf[pages=-,pagecommand={},width=1.3\textwidth]{Exhibits/doc.pdf}



\vspace*{\fill}
\begin{center}
{\LARGE \bf
Exhibit 23
}

\vspace{10\baselineskip}

{\large USNews article on top AI technology stocks}

\end{center}
\vspace*{\fill}


% \includepdf[pages=-,pagecommand={},width=1.3\textwidth]{Exhibits/US news top AI companies.pdf}



\vspace*{\fill}
\begin{center}
{\LARGE \bf
Exhibit 24
}

\vspace{10\baselineskip}

{\large Top venues in Artificial Intelligence, Mathematical Optimization, and Power Engineering}

\end{center}
\vspace*{\fill}


% \includepdf[pages=-,pagecommand={},width=1.3\textwidth]{Exhibits/Top Venues/Artificial Intelligence - Google Scholar Metrics.pdf}

% \includepdf[pages=-,pagecommand={},width=1.3\textwidth]{Exhibits/Top Venues/Journal Rankings on Artificial Intelligence.pdf}

% \includepdf[pages=-,pagecommand={},width=1.3\textwidth]{Exhibits/Top Venues/Mathematical Optimization - Google Scholar Metrics.pdf}

% \includepdf[pages=-,pagecommand={},width=1.3\textwidth]{Exhibits/Top Venues/Power Engineering - Google Scholar Metrics.pdf}



\vspace*{\fill}
\begin{center}
{\LARGE \bf
Exhibit 25
}

\vspace{10\baselineskip}

{\large Description of the Grid Modernization and the Smart Grid project by the Department of Energy}

\end{center}
\vspace*{\fill}

\includepdf[pages=-,pagecommand={},width=1.3\textwidth]{Exhibits/Grid Modernization and the Smart Grid _ Department of Energy.pdf}




\vspace*{\fill}
\begin{center}
{\LARGE \bf
Exhibit 26
}

\vspace{10\baselineskip}

{\large The White House Greenhouse Gas Pollution Reduction Target}

\end{center}
\vspace*{\fill}

\includepdf[pages=-,pagecommand={},width=1.3\textwidth]{Exhibits/the White House Greenhouse Gas Pollution Reduction Target.pdf}




\vspace*{\fill}
\begin{center}
{\LARGE \bf
Exhibit 27
}

\vspace{10\baselineskip}

{\large The description of Clean Energy Initiative by the State Energy Office }

\end{center}
\vspace*{\fill}

% \includepdf[pages=-,pagecommand={},width=1.3\textwidth]{Exhibits/doc.pdf}




\vspace*{\fill}
\begin{center}

{\LARGE \bf
Exhibit 28
}

\vspace{10\baselineskip}

{\large The page devoted to Artificial Intelligence (AI) by the U. S. Department of State}

\end{center}
\vspace*{\fill}

\includepdf[pages=-,pagecommand={},width=1.3\textwidth]{Exhibits/Artificial Intelligence (AI) - United States Department of State.pdf}




\vspace*{\fill}
\begin{center}

{\LARGE \bf
Exhibit 29
}

\vspace{10\baselineskip}

{\large New Actions to Promote Responsible AI Innovation that Protects Americans’ Rights and Safety}

\end{center}
\vspace*{\fill}

\includepdf[pages=-,pagecommand={},width=1.3\textwidth]{Exhibits/AI Biden-Harris Administration Announces New Actions to Promote Responsible AI Innovation.pdf}




\vspace*{\fill}
\begin{center}

{\LARGE \bf
Exhibit 30
}

\vspace{10\baselineskip}

{\large W-2 form M.E. Hu received from Anycompany Inc. for the calendar year 2022}

\end{center}
\vspace*{\fill}

% \includepdf[pages=-,pagecommand={},width=1.2\textwidth]{Exhibits/W2.pdf}





\end{document}

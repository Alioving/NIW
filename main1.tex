\documentclass{article}

% Language setting
% Replace `english' with e.g. `spanish' to change the document language
\usepackage[english]{babel}

% Set page size and margins
% Replace `letterpaper' with `a4paper' for UK/EU standard size
\usepackage[letterpaper,top=2cm,bottom=2cm,left=3cm,right=3cm,marginparwidth=1.75cm]{geometry}

% Useful packages
\usepackage{amsmath}
\usepackage{graphicx}
\usepackage[colorlinks=true, allcolors=blue]{hyperref}
\usepackage[final]{pdfpages}
\usepackage{fancyhdr}
\usepackage{setspace}
\usepackage[shortlabels]{enumitem}
\usepackage{lastpage}

\onehalfspacing
\setlength{\parskip}{0.5\baselineskip}%
\setlength{\parindent}{0pt}%

\renewcommand{\headrulewidth}{.0mm} % header line width

\pagestyle{fancy}
\fancyhf{}
\fancyhfoffset[L]{1cm} % left extra length
\fancyhfoffset[R]{1cm} % right extra length
% \rhead{\today}
\lhead{\small Permanent Residence Petition for Mr. Wenbin Hu}
\rfoot{Page \thepage ~of \pageref*{LastPage}}

% \title{Your Paper}
% \author{You}

\begin{document}
% \maketitle

% \begin{abstract}
% Your abstract.
% \end{abstract}
\vspace*{\fill}
\begin{center}

{\bf 
Immigrant Petition for Alien Worker\\
for the Alien with Exceptional Ability in Science (EB2-NIW)
}

\end{center}
\vspace*{\fill}

\begin{center}
TABLE OF CONTENTS
\end{center}
\begin{itemize}
    \item [p. \pageref*{G-1145}] Form G-1145 e-Notification of Application/Petition Acceptance. 
    \item [p. \pageref*{I-140}] Form I-140, Immigrant Petition for Alien Worker with the \$700 filing fee.
    \item [p. \pageref*{I-907}] Form I-907, Request for Premium Processing Service with the \$2805 filing fee.
    \item [p. \pageref*{docs}] Photocopies of the passports, F-1 visas, Form I-20, Form I-94, EAD cards.
    \item [p. \pageref*{IE}] Initial Evidence in Support of the I-140 Immigrant Petition.
    \item [p. \pageref*{plans}] Statement from Mr. Wenbin Hu detailing plans on how he intends to continue work in the United States.
    \item [p. \pageref*{exhib}] Exhibits 1–30.
\end{itemize}

\clearpage
\label{G-1145}
\includepdf[pages=-,pagecommand={},width=1.3\textwidth]{./forms n docs/g-1145.pdf}

\label{I-140}
\includepdf[pages=-,pagecommand={},width=1.3\textwidth]{forms n docs/i-140.pdf}

\label{I-907}
\includepdf[pages=-,pagecommand={},width=1.3\textwidth]{forms n docs/i-907.pdf}

\vspace*{\fill}
\begin{center}

{\LARGE \bf
Title Page of the Current Passport
}
\label{docs}
\end{center}
\vspace*{\fill}

 
% \includepdf[pages=-,pagecommand={},width=1.3\textwidth]{forms n docs/Int Pass 27 Front Page.pdf}

\vspace*{\fill}
\begin{center}

{\LARGE \bf
The Current Student Visa (F-1)\\
in an expired passport
}

\end{center}
\vspace*{\fill}


% \includepdf[pages=-,pagecommand={},width=1.3\textwidth]{forms n docs/F1 2023.pdf}

\vspace*{\fill}
\begin{center}

{\LARGE \bf
Title Page of the Expired Passport\\
and expired US visas
}

\end{center}
\vspace*{\fill}


% \includepdf[pages=-,pagecommand={},width=1.3\textwidth]{forms n docs/Int Pass 23.pdf}

\vspace*{\fill}
\begin{center}

{\LARGE \bf
Current form I-20
}

\end{center}
\vspace*{\fill}


% \includepdf[pages=-,pagecommand={},width=1.3\textwidth]{forms n docs/SignedSTEM_I-20.pdf}

\vspace*{\fill}
\begin{center}

{\LARGE \bf
Current form I-94
}

\end{center}
\vspace*{\fill}


% \includepdf[pages=-,pagecommand={},width=1.3\textwidth]{forms n docs/I94.pdf}

\vspace*{\fill}
\begin{center}

{\LARGE \bf
Current Employment Authorization Document
}

\end{center}
\vspace*{\fill}


% \includepdf[pages=-,pagecommand={},width=1.3\textwidth]{forms n docs/EAD 2023.pdf}

\vspace*{\fill}
\begin{center}

{\LARGE \bf
Expired Employment Authorization Document
}

\end{center}
\vspace*{\fill}

.

% \includepdf[pages=-,pagecommand={},width=1.3\textwidth]{forms n docs/EAD 2022.pdf}



\begin{flushright}
Wenbin Hu\\
999 99th st,\\
Default City, FL, 99999\\
Tel. (999) 999-9999
\end{flushright}

December 31, 2024

\label{IE}
% Premium Processing
% USCIS Nebraska Service Center
% P.O. Box 87103
% Lincoln, NE 68501-7103

USCIS Nebraska Service Center\\
850 S. Street, Lincoln, NE 68508

\underline{\bf Initial Evidence in Support of the I-140 Immigrant Petition}

\begin{tabular}{ll}
{\bf Petitioner and Beneficiary:} & Mr. Wenbin Hu \\
{\bf Classification Sought:} & Employment-Based Immigration, Second Preference, \\
& Exceptional Ability in Science \\
& with a “national interest waiver” of the job offer (EB2-NIW).\\
& Sec. 203(b)(2)(B) INA [8 U.S.C. 1153].
\end{tabular}
\vspace{2\baselineskip}

Dear USCIS Officer,

This initial evidence is submitted in support of Mr. Wenbin Hu, M.Eng. in Control Engineering, who is filing the I-140 Immigrant Petition for Alien Worker. The evidence demonstrates that Mr. Hu is an individual of exceptional ability in the field of science, specifically in Medical Artificial Intelligence, and that his work will substantially benefit the national economy, educational interests, and welfare of the United States in the future ({\it Please refer to Sections 1, 2 and 3}).

Mr. Hu provides evidence that he satisfies three (A, D, F) of six criteria listed in 8 CFR, Section 204.5(k)(3)(ii), namely:
\begin{enumerate}
    \item Mr. Hu has an advanced degree in Control Engineering from a university in China. ({\it Please refer to Sections 1.1 and 3.2})
    \item Mr. Hu has commanded remuneration for his services, which demonstrates exceptional ability; ({\it Please refer to Section 1.2})
    \item Evidence of Mr. Hu's recognition for achievements and significant contributions to the field by peers and professional organizations. ({\it Please refer to Section 1.3})
\end{enumerate}
Due to the specifics of the highly-competitive area of Mr. Hu's occupation, Mr. Hu additionally provides evidence that he satisfies the following two (iv, vi) of ten criteria listed in 8 CFR, Section 204.5(h)(3) for determination of the extraordinary abilities. 
\begin{enumerate}
    \item Mr. Hu has participated, both individually and on a panel, as a judge of the work of others in the fields of Control Systems, Electrical Engineering, and Artificial Intelligence. ({\it Please refer to Section 1.4})
    \item Mr. Hu has authored scholarly articles in the fields of Computational Neuroscience, Cognitive Systems, and Artificial Intelligence. ({\it Please refer to Section 1.5})
\end{enumerate}

The criteria listed in 8 CFR, Section 204.5(h)(3) are comparable to the criteria listed in 8 CFR, Section 204.5(k)(3)(ii) due to the standards of exceptional ability being lower than the standard for extraordinary ability classification.

Mr. Hu is seeking a national interest waiver of the job offer, as per 8 USC 1153(b)(2)(B)(i) and 8 CFR  204.5(k)(4)(ii). The attached evidence and statement satisfy all three criteria for such a waiver, as described in Matter of Dhanasar, 26 I\&N Dec. 884 (AAO 2016). The supporting documentation and statement are as follows:

\begin{enumerate}
    \item Mr. Hu's proposed work in Medical Artificial Intelligence has both substantial merit and national importance. ({\it Please refer to Section 2})
    \item Mr. Hu is well-positioned to advance the proposed endeavor due to his expertise. ({\it Please refer to Sections 1 and 3})
    \item On balance, it would be beneficial to the United States to waive the job offer and labor certification requirements for Mr. Hu. ({\it Please refer to Sections 2 and 4, and Statement from Mr. Hu detailing plans on how he intends to continue work in the United States})
\end{enumerate}

In the United States, Mr. Hu plans to continue to work in the area of expertise. ({\it Please refer to the Statement from Mr. Hu detailing plans on how he intends to continue work in the United States and to Exhibits 8 and 9, his current job offers})

Pursuant to 8 CFR, Section 204.5(k)(1), M.E Hu may file a petition on Form I-140 for classification under Section 203(b)(2) of the Act as an alien of exceptional ability in the sciences on his own behalf because he is seeking an exemption from the requirement of labor certification in the United States pursuant to Section 203(b)(2)(B) of the Act.

\clearpage

{\bf \underline{Section 1.} Mr. Hu is an alien of exceptional ability in the Medical Artificial Intelligence who will prospectively substantially benefit the national economy, educational interests, and welfare of the United States.}

Mr. Hu's main area of study focuses on addressing a significant and pervasive health challenge: epilepsy and the unpredictable seizures it causes. By leveraging state-of-the-art computational techniques to predict seizures more accurately, the Beneficiary's research aims to enable proactive patient care, reduce the burden on the U.S. healthcare system, cut costs, improve quality of life for patients, and maintain the United States’ global leadership in medical innovation and biomedical engineering research. His planned endeavor is based on top of his past work and aims to advance these sub-fields even further.

 
{\bf 1.1 Mr. Hu has received degrees, including a B.Sc. degree in Electrical Engineering and Automation and a Master’s degree in Control Engineering  from high-ranking Universities. }

Mr. Hu obtained his Bachelor Degree of Electrical Engineering and Automation and Master Degree of Control Engineering from the Hangzhou Dianzi University ({\it Exhibit 8, Bachelor degree of Electrical Engineering and Automation; Exhibit 9, Master Degree of Control Engineering}). According to the ARWU Ranking 2024, it was near the top 3\%  best university in China (97/3074) and one of the Top 600 universities in the world. Specifically for the control automation professional ranking, the university is ranked top 150 ({\it Exhibit 10, Academic Ranking of World Universities 2024; Exhibit 11, Statistical Bulletin on the Development of National Education in China}). 

His graduate-level performance was exceptional too, achieving a high GPA (89.9/100) and ranking 4th out of 172 students, culminating in an “A” rating for his Master’s dissertation ({\it Exhibit 12, Graduate transcript}). These academic achievements prove that he has mastered the complex theoretical framework of machine learning, signal processing, and artificial intelligence application principles.

{\bf 1.2. Mr. Hu has commanded a high compensation for services, demonstrating exceptional ability. }

When the applicant first graduated, the applicant was awarded an monthly salary of 17000 yuan at Huawei Technologies Co., Ltd. for his outstanding research. ({\it Exhibit 13, Bank Revenue Flow 2019.}) Huawei is a top company in China and enjoys a certain reputation in the world. And compared to the average salary of a fresh graduate in 2019, Mr. Hu's salary is 2.5 times of computer engineer (6,858 yuan/month), which is the highest paid in all field ({\it Exhibit 14, Salary of China's fresh graduates sees steady growth: report}).

%https://english.www.gov.cn/news/topnews/202007/10/content_WS5f082574c6d06c4091250b16.html
After five years of employment, Mr. Hu was recommended by the company to work in Hungary, where his total compensation exceeded 600,000 yuan. This total income is comprised of three components: the salary paid by the local Hungarian subsidiary  ({\it Exhibit 14, Subsidiary revenue proof, the currency is HUF, equal to xx yuan}), the salary and benefits provided by the parent company ({\it Exhibit 15, Proof of Parent company's Income in 2024}), and a stock dividend of 60,000 yuan ({\it Exhibit 16, Certificate of dividend income after tax, excluding 20 percent tax}). The applicant's earnings are three times the average salary in the computer field for individuals with the same five years of experience, and significantly exceed the compensation in the highest-paying positions within the industry ({\it Exhibit 16, Average survey for five years experiences employee in 2024}).

The illustrated major difference in Mr. Hu's compensation from the job market's standards comes from the competitive advantage that Mr. Hu has in the China job market due to his exceptional abilities in his fields of expertise. 

{\bf 1.3 Mr. Hu is recognized for achievements and significant contributions to the fields of Epileptic Seizure Prediction and Machine Learning by peers and professional organizations. }

{\bf 1.3.1 Other scientists recognize Mr. Hu’s exceptional knowledge of Epileptic Seizure Prediction and Machine Learning and consider Mr. Hu a top expert in these fields.}

Mr. Hu's international recognition is evident from the 6 letters supporting his petition that he received from six distinguished professionals from the China and abroad. ({\it Supporting Letters; Exhibits 2–7.}) 

All authors of supporting letters are recognized experts in the fields of Epileptic Seizure Prediction or Machine Learning. Three of them have been Mr. Hu’s mentors, while the other three have never worked with Mr. Hu directly but know his work from his publications and collaborative projects. 

“What sets Hu apart is his rare blend of technical expertise and applied problem-solving acumen. This unique skill set, which goes beyond standard engineering or data science competencies, is not easily replicated. The United States stands to gain substantially from his immediate and unrestricted involvement in ongoing research endeavors, clinical collaborations, and the development of next-generation medical devices.” ({\it Exhibit 4, letter from Mr. Xiangshao Liu})

“Through sophisticated CNN architectures and multi-layered deep learning frameworks, he has significantly improved the sensitivity and specificity of seizure prediction models we are developing. This improvement is not a marginal enhancement—it can fundamentally change clinical approaches to epilepsy management. By enabling preemptive intervention and careful resource allocation, these methods align seamlessly with U.S. objectives to enhance preventive care, mitigate chronic disease burdens, and reduce the overall financial strain on our healthcare system.” ({\it Exhibit 5, a letter from Professor D})

“The capacity to predict seizures before they occur is a transformative goal for neurology. By reducing unpredictability, we can lower emergency admission rates, refine treatment schedules, and empower patients with greater autonomy. The United States has invested extensively in health technologies that promote prevention, cost-effectiveness, and improved patient outcomes. Hu’s contributions exemplify these priorities, moving beyond traditional algorithms to extract previously inaccessible predictive insights from complex EEG signals.” ({\it Exhibit 7, a letter from Professor E.})

“Epilepsy management is an enduring challenge that consumes significant healthcare resources and often results in acute patient distress. By employing CNN-based deep learning methods to identify preictal EEG patterns with unprecedented accuracy, Hu has expanded our toolkit for anticipatory patient care. This improvement is no small feat. It reflects the kind of breakthrough that advances U.S. national interests in reducing healthcare expenditures, improving patient safety, and strengthening our capabilities in precision medicine.” ({\it Exhibit 6, a letter from Dr. D.})

{\bf 1.3.2 Mr. Hu has made original discoveries in Epileptic Seizure Prediction or Machine Learning. }

“Hu’s work in deploying stacked CNN architectures and refined feature extraction methods has substantially advanced our collective ability to forecast seizures. This step forward aligns seamlessly with national interests, as the U.S. healthcare system continually seeks innovative tools to improve patient care, minimize acute interventions, and contain escalating costs. His research directly supports these aims by turning complex EEG data into actionable clinical intelligence.” ({\it Exhibit 2, a letter from Professor Jiuwen Cao}) 

“[...] Hu has made highly distinctive contributions to this field by integrating sophisticated Convolutional Neural Network (CNN) architectures, including stacked deep learning frameworks, to identify subtle preictal EEG biomarkers. His work, substantiated by peer-reviewed publications, not only surpasses conventional analytical approaches but also establishes new benchmarks for predictive accuracy and reliability.” ({\it Exhibit 3, a letter from Ms. Yibing Shen.}) 

{\bf 1.3.3 Many scientists highly cite the papers co-authored by Mr. Hu. }

The significance and the impact of Mr. Hu’s work are demonstrated by the fact that his papers have been cited at least 187 times by xx research groups from the United States and other countries, according to citation reports from the Google Scholar citation database. ({\it Exhibit 18, Citation statistical results}) This number is constantly growing at rates higher than the impact factors of some of the corresponding journals. It is impressive for a young scientist who published his first paper merely 5 years ago when he was an graduate student. 

Mr. Hu’s papers have been cited by Professor H, the Vice Chancellor for Research and Distinguished Professor of Electrical Engineering and Computer Science at the University of H, by Prof. J, a professor of Machine Learning in the Computational and Biological Learning Lab, Department of Engineering, University of J, and Prof. K, a top researcher on Optimization in Machine Learning and AI, the Moorthy Family Professor in the departments of Mathematics, Statistics, and the Allen School of Computer Science and Engineering at the University of K.

Hu’s work has been widely recognized by the academic and professional communities for its groundbreaking nature and significant impact on the field. Numerous studies have cited Hu's research as a key reference, acknowledging its innovative methodologies and foundational insights. These acknowledgments highlight the critical role his work has played in advancing both theoretical understanding and practical applications within the domain.

“In this study, frequency domain features and spatial domain features are selected as important indicators of landscape recognition. The frequency domain features include: mean amplitude spectrum (MAS) [45], ..." ({\it Exhibit 19, Citation reports for Mr. Hu’s papers})

“In our model, our fragment sampling adopts the method of [12,17], that is, 50\% overlapping sampling of the data is performed to obtain the sampled data ... " ({\it Exhibit 19, Citation reports for Mr. Hu’s papers})

“Hu et al. [2] and Cao et al. [10] classified interictal, preictal-I, preictal-II, preictal-III, and ictal courses of 5-class by conducting cross-validation experiments. The comparison drawn between the proposed model and the above-mentioned works is presented in the 5-class cross-validation part of Table VI. The confusion matrices of the state-of-the-art model and the proposed model are illustrated in Fig. 6(a) and (b), respectively, which show that the accuracy of the preictal-I, preictal-II, and preictal-III category of the proposed model is significantly better than the state of the art with an improvement of 8.5\%, 12.6\%, and 9.3\%, respectively. In summary, the proposed HGCN+TC+PF model shows a 5.77\% improvement in accuracy compared with the state of the art." ({\it Exhibit 19, Citation reports for Mr. Hu’s papers})

“Furthermore, similar to [2], each of $\delta$, $\theta$, and $\alpha$ bands are subdivided into 3 sub-bands of equal size, and each of the last two bands are subdivided into 5 sub-bands of equal size." ({\it Exhibit 19, Citation reports for Mr. Hu’s papers})


Moreover, many researchers have extended Hu’s contributions by leveraging his methodologies, such as adopting his algorithmic frameworks, applying his feature extraction techniques, and using his results for benchmarking and comparative studies. In several cases, researchers have replicated his approaches to validate their effectiveness, while others have refined his methods to address new challenges. This widespread adoption and adaptation of Hu’s work demonstrate its versatility and enduring influence in the field.

“Cao et al. proposed algorithm yields superior results than other existing algorithms. " ({\it Exhibit 19, Citation reports for Mr. Hu’s papers})


“Previous research has shown promising results using convolutional neural networks (CNNs) for epileptic seizure prediction. For example, Hu et al. [22] used a CNNmodel and  achieved an accuracy of 86.25\% on the CHB-MIT dataset. The model demonstrated an improvement of nearly 12\% in accuracy for identifying preictal samples closer to a seizure, compared to those further in time. " ({\it Exhibit 19, Citation reports for Mr. Hu’s papers})

“Considering related studies, the Convolutional Neural Network (CNN) model of Hu et al. [22] achieved 86.25\% accuracy on the CHB-MIT-EEG dataset. Their approach demonstrated an almost 12\% accuracy improvement when identifying pre-ictal samples closer to a seizure compared to those more distant in time." ({\it Exhibit 19, Citation reports for Mr. Hu’s papers})

“The discriminative features are extracted from the raw scalp EEG (sEEG) to predict the seizure epochs effcaciously by adopting CNN having sensitivity of 83.33\% [48], prediction accuracy of 90.0\% [49], and 99.33\% using stacked CNN [50]. " ({\it Exhibit 19, Citation reports for Mr. Hu’s papers})

“With a large amount of data, it outperforms traditional feature extraction in terms of classification accuracy (Hu et al., 2019). "  ({\it Exhibit 19, Citation reports for Mr. Hu’s papers})

“Traditional manual feature extraction methods often  suffer from low generalization and suboptimal performance. To  overcome these limitations, researchers have increasingly integrated  manual feature extraction with deep learning techniques (Cao et al.,  2019; Yuan et  al., 2018). This combined approach leverages the 
strengths of both methods, enhancing model performance and  accuracy in epileptic seizure detection and prediction. "  ({\it Exhibit 19, Citation reports for Mr. Hu’s papers})

“Artificial neural networks  can collect information on samples, generalize, and then decide on those samples using learned information compared to samples they have never seen before. Due to these learnings and generalizations, artificial neural networks find wide applications in many scientific fields and demonstrate their ability to solve complex problems successfully [6]. " ({\it Exhibit 19, Citation reports for Mr. Hu’s papers})

“Other studies [23–27] used the raw EEG signals without any preprocessing and also achieved good results in predicting epileptic seizures. Besides, Cao and Hu et al. [28–31] achieved multi level prediction of epilepsy and obtained good results using the Mean  Amplitude Spectrum (MAS)."  ({\it Exhibit 19, Citation reports for Mr. Hu’s papers})

“Cao et al. proposed a method to predict seizures using a stacked CNN (SCNN) model composed of different CNN structures, characterized by the
average amplitude spectrum map (MAS) of the $\alpha$, $\beta$, $\theta$ and $\gamma$ rhythms of sEEG signals. They applied an adaptive feature weighting fusion algorithm to the classification process, which greatly enhanced the performance of the model [20]. "   ({\it Exhibit 19, Citation reports for Mr. Hu’s papers})

“Much success has been recorded due to its ability to deal with a large amount of data and learn from the raw data, eliminating the need for hand feature extraction as in conventional techniques (Hu et al., 2019; Malekzadeh et al., 2021)."  ({\it Exhibit 19, Citation reports for Mr. Hu’s papers})

“Deep learning algorithm is more effective in processing large and complex bioelectric signals, so it has been gradually utilized in the field of epilepsy detection in recent years, especially the Convolutional Neural Network (CNN) (Hu et al., 2019). "  ({\it Exhibit 19, Citation reports for Mr. Hu’s papers})

Hu’s contributions have served as a cornerstone for advancements in his area of expertise, underpinning significant progress in both academic research and industrial applications. The widespread recognition, adoption, and adaptation of his work underscore its importance to the broader community. These factors collectively illustrate that Hu’s continued efforts align with the national interest, making a compelling case for his qualifications under the National Interest Waiver (NIW) criteria.


{\bf 1.4. Mr. Hu has been a judge of the work of others in the fields of Machine Learning and Artificial Intelligence. }

Mr. Hu has completed two peer review assignments from Journal of the Franklin Institute and AI MED journals, and has received invitations from two other journals, namely International Joint Conference on Neural Networks(IJCNN) 2025 and Association for the Advancement of Artificial Intelligence(AAAI) 2025. These are famous SCI journals and conferences in the field of artificial intelligence and engineering, for example, AAAI Conference on Artificial Intelligence is the \#4 venue in the ranking for Artificial Intelligence, which shows that hu's achievements are recognized by many journals.({\it Exhibit 20, Review assignments completed by Mr. Hu and review invatation by IJCNN and AAAI; \it Exhibit 21, Journal Ranking })


“[...] we have requested Mr. Hu’s expertise in the area of medical artificial intelligence under uncertainty to review the paper independently and provide feedback to the authors. The review was completed swiftly. Mr. Hu has made valuable suggestions allowing us to accept the manuscript after a revision, delivering an impact to the applications such as signal recovery with uncertainties in the sensing matrix and identification of parameters of time-invariant discrete-time linear dynamical systems via noisy observations.” ({\it Exhibit 22, A letter from Dr. XX, Editor of the AI MED jorunal}) 


{\bf 1.5 Mr. Hu is widely published in the fields of Computational Neuroscience and Artificial Intelligence. His publications have appeared in top journals in these fields.}

Mr. Hu has already published 2 peer-reviewed articles, and he also has a paper in the works, waiting to be published for peer review. ({\it Exhibit 23, First pages of 3 papers co-authored by Mr. Hu}) 

“Mr. Hu’s thesis results are published in prestigious venues as he made key experimental and intellectual contributions to many results produced in my group. Under my supervision, he has published 2 journal articles because of his key intellectual contribution to the work. Both of his journal articles are published in the top journals in the fields of Artificial Intelligence (Journal of Ambient Intelligence and Humanized Computing , top-xx h-index among open access journals) and Computational Neuroscience (IEEE Transactions on Cognitive and Developmental Systems, top-xx h-index overall).” ({\it Exhibit 2, a letter from Professor Cao}) 

“The work with us is only a very small part of the Mr. Hu’s record since he has co-authored 2 papers before starting his work at our company. That is an impressive work, which summarizes well Mr. Hu’s commitment to his research. I consider myself very fortunate that he accepted to work on our research project.” ({\it Exhibit 3, Letter of Ms. Sheng})

According to the Google Scholar Metrics rankings of the venues in Artificial Intelligence and Computational Neuroscience, the journals and conferences that published papers by Mr. Hu are at the top of these fields by H-factor in 2023. ({\it Exhibit 24, Top venues rankings.}) For example, Journal of Ambient Intelligence and Humanized Computing is the top xx venue for publications on Artificial Intelligence, IEEE Transactions on Cognitive and Developmental Systems is the top-xx among the venues on Computational Neuroscience.

{\bf 1.6 Mr. Hu has performed a critical role in projects carried out for organizations of distinguished reputation. His knowledge and contribution to the fields have been invaluable to this job and have greatly impacted his area of study. }

“(From Cao)xx played an important role in the research team, not only completed its own research content outstandingly, but also completed the research beyond expectations with other team members, playing an indispensable role in the team.” ({\it Exhibit 2, a letter from Professor Cao}) 

“Besides this paper, Mr. Hu co-authored several other high-profile papers during his work as a master student with his adviser Prof. Cao and colleagues from the Hangzhou Dianzi University. This distinction means that he was a key person in the published research and carried out most of the intellectual work.” ({\it Exhibit 7, a letter from Dr. C}) 

"(From Liu)Mr. Hu performs his duties as a engineer responsible for algorithms and software development at the AI+ Business Application Delivery Department of Huawei Technologies Co., Ltd ({\it Exhibit 25, Employee Certificate of Incumbency}). Huawei is one of the top Artificial Intelligence technology companies in the China ({\it Exhibit 26, 12 Leading Chinese Companies in the AI Large Model Sector}). His responsibilities include researching machine learning models for predicting user behavior and intelligent conversations."

“(From Sheng)The optimization procedure was distributed over thousands of computational machines that had to work synchronously for many months, making the experiment extremely difficult to conduct from a technological point of view. This is where Mr. Hu applied his expertise, becoming responsible for the resiliency and robustness of the computational load to the potential issues coming from the unreliability of the underlying hardware and the limitations of the existing optimization algorithms used in modern AI training. Mr. Hu was surprisingly fast to onboard with the new team and worked collaboratively with our engineers which led to the successful completion of the planned experiments. [...] Working in our group, John has performed a key function and obtained hands-on experience in training large machine learning models that many believe to be the future of AI and the key to the next industrial revolution.” ({\it Exhibit 3, Letter of Ms. Sheng}) 


\clearpage


{\bf \underline{Section 2.} Mr. Hu’s proposed employment has both substantial merit and national importance for the United States.}

Mr. Hu intends to work in the fields of Epileptic Seizure Prediction and Machine Learning. The following descriptions of federal and state projects provide a summary of the importance and the urgency of the research in  Seizure Prediction:

{\bf 2.1 Epilepsy in the United States: A Pressing Public Health Challenge. }

Epilepsy affects approximately 3.4 million Americans, including nearly 3 million adults and 470,000 children, according to the Centers for Disease Control and Prevention (CDC). Characterized by recurrent seizures that occur unpredictably, epilepsy can result in injuries, hospitalizations, social stigma, and reduced quality of life. Patients and families often must navigate unpredictable episodes, impacting employment, education, and social integration. ({\it Exhibit 27, Epilepsy Facts and Stats-According to CDC}) 


Beyond the human toll, epilepsy imposes a significant economic burden. Healthcare costs related to epilepsy run into billions of dollars annually, including emergency room visits, hospital stays, long-term care costs, medications, and indirect costs related to lost productivity and disability. Interventions that reduce seizure frequency, severity, or unpredictability have enormous potential to alleviate these burdens  ({\it Exhibit 27, Epilepsy Facts and Stats-According to CDC; \it Exhibit 28, H.R.10210 - National Plan for Epilepsy Act})


{\bf 2.2 Why Seizure Prediction Matters. }

The unpredictability of seizures is a central challenge. Currently, treatments focus on reducing seizure frequency or severity through medications, surgical interventions, or devices like vagus nerve stimulators. But the unpredictable nature of seizure onset often remains. Patients cannot reliably know when a seizure will occur, which elevates risk, anxiety, and sometimes necessitates continuous care or supervision.

Accurate seizure prediction would be transformative. If patients could receive reliable warnings minutes or even seconds before a seizure, they could move to a safe space, prepare emergency medication, or contact support. Clinicians could personalize treatments based on an individual’s EEG profile, adjusting medication dosages or scheduling therapeutic interventions more effectively. This shift from reactive to proactive management aligns closely with U.S. healthcare goals emphasizing preventive care and patient empowerment.

{\bf 2.3 Advancing Epilepsy Diagnosis and Monitoring Through Artificial Intelligence-Based EEG Analysis. }

EEG is a cornerstone tool in epilepsy diagnosis and monitoring. EEG signals capture electrical activity in the brain, providing a non-invasive way to detect abnormal patterns that may herald a seizure. However, raw EEG data is notoriously complex, noisy, and high-dimensional. Human experts can interpret certain patterns, but subtle preictal states are often too subtle or variable to detect with conventional methods.

Machine learning, and deep learning in particular, excel at finding complex patterns in large datasets. CNNs have revolutionized image classification and speech recognition, and now they are being applied successfully to time-series biomedical data. By training CNNs on EEG recordings of patients who have seizures, these models learn to identify signatures that precede seizures. This leads to better accuracy, fewer false alarms, and earlier warnings.

{\bf 2.4 The Beneficiary’s Two Publications and Their Contribution to National Importance. }

{\bf First Publication in IEEE Transactions on Cognitive and Developmental Systems (2019):}

The Beneficiary introduced stacked CNNs to classify epileptic EEG signals, capturing complex temporal and spatial dynamics. This innovation significantly improved the accuracy, sensitivity, and specificity of seizure detection, advancing deep learning techniques for clinical applications. Published in a highly respected journal, this work laid a strong foundation for AI-driven seizure monitoring, addressing a critical need in U.S. healthcare to improve epilepsy diagnosis and patient care ({\it Exhibit 23, First pages of 3 papers co-authored by Mr. Hu}).

{\bf Second Publication in Journal of Ambient Intelligence and Humanized Computing (2023):”}

Building on this, the Beneficiary developed a mean amplitude spectrum-based method for epileptic state classification using CNNs. By isolating key frequency-domain features, this approach enhanced seizure prediction reliability, facilitating earlier intervention. This research directly supports U.S. efforts to integrate AI into clinical practice, improving outcomes for epilepsy patients and reducing healthcare burdens ({\it Exhibit 23, First pages of 3 papers co-authored by Mr. Hu}). 

{\bf Third Publication (Under Preparation): “Real-World Validation of AI-Based Algorithms for Epileptic Seizure Prediction Using Hospital EEG Data”}

Besides, the Beneficiary is still actively contributing as a member of Professor Cao's research group, where we continue to integrate the latest technologies into epileptic seizure detection. In collaboration with hospitals, we are applying advanced algorithms to real-world hospital EEG data to improve seizure prediction and patient outcomes. The Beneficiary's latest research proposes a novel algorithm based on [insert specific method, e.g., "attention-based deep learning models" or "hybrid CNN-RNN frameworks"], which further enhances the accuracy and timeliness of epileptic seizure detection. This work not only validates the effectiveness of the Beneficiary’s achievements on real-world clinical data but also demonstrates their potential to significantly improve healthcare delivery ({\it Exhibit 23, First pages of 3 papers co-authored by Mr. Hu}).

These techniques can be seamlessly applied to U.S. medical institutions, research centers, and hospitals, providing economic and practical benefits by reducing the cost of prolonged EEG monitoring, minimizing false alarms, and enabling timely intervention for epilepsy patients. Moreover, this research addresses a pressing public health challenge in the United States, where epilepsy affects over 3 million individuals. By advancing AI-driven seizure prediction tools, the Beneficiary’s work directly contributes to improving patient care, optimizing resource utilization, and fostering innovation in U.S. medical technology sectors, aligning with national priorities in healthcare and AI development.

{\bf 2.5 Alignment with U.S. National Health and Research Priorities. }

The United States invests heavily in neurological research, often through NIH grants and other federal funding mechanisms. The NIH’s NINDS (National Institute of Neurological Disorders and Stroke) supports studies that improve diagnostics and treatments for epilepsy. ({\it Exhibit 29, Focus On Epilepsy Research-NINDS.}) Meanwhile, the NIH BRAIN Initiative encourages developing innovative tools and analytical methods to understand brain activity better and treat neurological conditions ({\it Exhibit 30, Tools and Technologies for Brain Cells and Circuits-BRAIN}).

The Beneficiary’s research direction—a stronger, more precise method of EEG-based seizure prediction—resonates with these federal priorities. By improving prediction, we move closer to precision medicine in neurology, where treatments and interventions are tailored based on data-driven insights. This synergy with national strategies ensures that the Beneficiary’s work is not an isolated academic exercise, but directly relevant to ongoing federal and institutional efforts to advance neurological care.


{\bf 2.6 Economic and Strategic Implications. }

From an economic standpoint, breakthroughs in seizure prediction offer a path toward reducing the overall cost of epilepsy on the U.S. healthcare system. As healthcare models shift toward value-based care, interventions that prevent costly ER visits, hospitalizations, and complications align perfectly with national cost-containment and efficiency goals.

The United States is strategically committed to maintaining its leadership in medical AI and biomedical device innovation. Enhanced seizure prediction technologies, particularly those utilizing EEG data, have the potential to drive the growth of American startups developing EEG-based wearables, telemedicine platforms, and hospital decision-support systems. These advancements can lead to job creation, attract global customers, and reinforce the U.S. as a hub of medical technology innovation.

The U.S. Department of Health and Human Services (HHS) has articulated an Artificial Intelligence (AI) Strategy aimed at leveraging AI capabilities to solve complex mission challenges and generate AI-enabled insights, thereby removing barriers to AI innovation. This strategy aligns with the national commitment to promote the use of trustworthy AI and maintain American leadership in AI  ({\it Exhibit 31, Artificial Intelligence (AI) Strategy-HHS}).

Furthermore, the National Artificial Intelligence Research and Development Strategic Plan emphasizes the importance of advancing AI research and development to ensure continued U.S. leadership in AI. This plan highlights the need for innovation in AI technologies, which includes applications in medical devices and healthcare systems. ({\it Exhibit 32, National Artificial Intelligence Research and Development Strategic Plan 2023}).

The integration of AI into medical devices, such as EEG-based wearables, is already underway. For instance, companies are developing wearable health devices that support real-time diagnostics, continuous monitoring, and immediate feedback, ensuring patients receive timely interventions without visiting a clinic ({\it Exhibit 33, From smart shoes to EEG-13 companies pioneering wearable medical devices-ESCATEC}).

These developments underscore the significant economic and practical benefits that advancements in seizure prediction technologies can bring to the United States, extending well beyond direct patient care.

{\bf 2.7 Humanitarian and Ethical Dimensions. }

Improving epilepsy care is indeed a humanitarian imperative, as individuals with epilepsy often face stigma and limitations that impede their independence. In the United States, several policies and initiatives aim to address these challenges, promoting autonomy and equal opportunity for those affected:

Americans with Disabilities Act (ADA): The ADA prohibits discrimination against individuals with disabilities, including epilepsy, in various areas such as employment, public services, and accommodations. It mandates reasonable accommodations in the workplace to ensure equal opportunities. For example, the U.S. Equal Employment Opportunity Commission (EEOC) provides guidance on workplace rights for people with epilepsy, emphasizing that employers must provide reasonable accommodations and cannot discriminate based on an individual's condition ({\it Exhibit 34, Epilepsy in the Workplace and the ADA}).

Epilepsy Foundation Initiatives: Organizations like the Epilepsy Foundation actively work to reduce stigma and empower individuals with epilepsy. They provide resources and advocate for policies that support the rights and well-being of those affected. Their efforts include public education campaigns to dispel myths about epilepsy and promote understanding.

Epilepsy Leadership Council (ELC) Policy Agenda: The ELC supports policies aimed at reducing health disparities and promoting health equity within the epilepsy community. Their agenda focuses on increasing awareness, reducing stigma, and advocating for better access to care and resources for individuals with epilepsy ({\it Exhibit 35, The Epilepsy Leadership Council (ELC) Policy Agenda}).

These policies and initiatives align with American values of empowerment and equal opportunity, striving to enhance the quality of life for individuals with epilepsy by promoting autonomy, reducing stigma, and ensuring equal access to opportunities.

In summary, the national importance of the Beneficiary’s research is evident at multiple levels: public health improvements, alignment with federal research priorities, economic benefits, strategic technological leadership, and ethical considerations that enhance patient well-being. These intersecting factors firmly anchor the Beneficiary’s work as having substantial merit and national importance.


\clearpage

{\bf \underline{Section 3.} The Beneficiary is well-positioned to advance the proposed endeavor due to his expertise. }

The second Dhanasar prong necessitates that the Beneficiary be well-positioned to advance his endeavor. This means he must have the skills, experience, credibility, and track record to move his research from theory to impactful application. The Beneficiary amply meets these criteria.


{\bf 3.1 Academic Excellence and Engineering Foundations.}

The Beneficiary’s advanced degree (Master’s in Control Engineering) from Hangzhou Dianzi University and high GPA highlight his capacity to master complex concepts. Control Engineering integrates mathematics, systems analysis, and signal processing—the perfect substrate for learning and implementing EEG data analytics. Ranking 4 out of 172 peers and earning an “A” dissertation rating indicate that his professors and mentors recognized his abilities, analytical rigor, and research potential ({\it Exhibit 12, Graduate transcript}).
Such technical and theoretical competence is crucial, as EEG-based seizure prediction involves intricate pattern recognition, understanding signal theory, statistical modeling, and advanced algorithmic design. Without a strong theoretical grounding, it would be challenging to navigate the complexity of EEG signals and deep learning frameworks ({\it Exhibit 1, CV of Mr. Hu}) .

{\bf 3.2 Comprehensive Research Experience in EEG-based Seizure Prediction.}

During his time in Professor Cao’s group, the Beneficiary engaged in full-stack EEG analytics. He did not specialize in just one facet; he learned to handle raw EEG signals, reduce noise, extract critical features, and implement machine learning pipelines. This hands-on immersion provided a holistic understanding of the challenges and solutions in epilepsy research.

He explored different machine learning models: CNN, SVM, KNN, and weighted fusion strategies. Such breadth ensured that he could compare models, tune hyper-parameters, and integrate multiple architectures for optimal results. This versatility sets him apart from researchers who know only a single modeling approach. The Beneficiary can tailor solutions to diverse data sets and clinical conditions, a vital skill as no “one-size-fits-all” model exists for all EEG variations.


{\bf 3.3 Peer-Reviewed Publications Demonstrating Ongoing Innovation.}

{\bf First Publication in IEEE Transactions on Cognitive and Developmental Systems (2019):}

The Beneficiary introduced stacked CNNs to classify epileptic EEG signals, capturing complex temporal and spatial dynamics. This innovation significantly improved seizure detection accuracy, sensitivity, and specificity, laying a strong foundation for AI-driven clinical applications. ({\it Exhibit 23, First pages of 3 papers co-authored by Mr. Hu.})

{\bf Second Publication in Journal of Ambient Intelligence and Humanized Computing (2023):}

Building on this, the Beneficiary developed a mean amplitude spectrum-based method to enhance seizure prediction reliability. This work facilitates earlier interventions, aligning with U.S. goals to integrate AI into clinical practice and reduce healthcare burdens. ({\it Exhibit 23, First pages of 3 papers co-authored by Mr. Hu.})

{\bf Third Publication (Under Preparation): }

Currently, the Beneficiary collaborates with hospitals to validate advanced algorithms on real-world EEG data. His latest method, based on [insert specific technique], further improves prediction accuracy and timeliness, demonstrating its practical impact on clinical care. This research bridges theoretical advancements and real-world implementation, addressing a pressing U.S. public health challenge. ({\it Exhibit 23, First pages of 3 papers co-authored by Mr. Hu.})

Mr. Hu is a talented and dedicated researcher whose contributions to AI-driven seizure prediction are groundbreaking. His work improves diagnostic accuracy, facilitates earlier intervention, and supports U.S. leadership in medical AI innovation. I am confident his continued efforts will have lasting benefits for patients and the healthcare system alike ({\it Exhibit 4, a letter from Professor Cao}).

{\bf 3.4 Patent Application: Moving Toward Applied Innovations.}

His contribution to a Chinese patent application (No. 201810116608.9) related to CNN-based pre-seizure detection methods signals a transition from theory to innovation. Patents underscore novelty, utility, and potential commercial viability. By developing intellectual property, he shows readiness to contribute to medical device manufacturing, software solutions, and other commercial channels that could deliver immediate benefits to U.S. healthcare systems upon his involvement ({\it Exhibit 36, Chinese patent authored by Mr. Hu}).

{\bf 3.5 Technical Expertise and Adaptability for Cutting-Edge AI Research.}

"The Beneficiary possesses exceptional technical skills that make him an asset to any U.S. lab or R\&D environment. His expertise in Python, data analysis libraries, Linux systems, and frameworks like TensorFlow enables him to adapt quickly to modern medical research workflows. He is highly proficient in building and optimizing TensorFlow models, running large-scale EEG data analyses efficiently, and ensuring thorough documentation of results. Additionally, his familiarity with open-source software stacks, version control systems, and reproducible machine learning pipelines aligns perfectly with the standards of high-level research collaborations. These capabilities demonstrate his readiness to contribute meaningfully to cutting-edge AI-driven medical research projects." ({\it Exhibit 4, Letter of Mr. Liu})


{\bf 3.6 Interdisciplinary Expertise and Collaborative Potential.}

Epilepsy research is inherently multidisciplinary, blending neurology, clinical practice, biomedical engineering, signal processing, and machine learning. The Beneficiary’s profile-combining engineering foundations with deep learning advancements—makes him a valuable addition to U.S. teams that may already include neurologists, clinicians, data scientists, and hardware developers. He can communicate effectively across disciplines, aligning technical solutions with clinical realities.

{\bf 3.7 Leveraging U.S. Research Infrastructure to Advance AI-Driven Epilepsy Solutions.}

Once in the U.S., the Beneficiary could partner with NIH-funded epilepsy research consortia or collaborate with major institutions that have large EEG datasets. His methods, refined on these larger and more diverse data sets, could improve generalizability. This environment would let him iterate quickly, produce new models, and test them in clinical pilot studies. Such a fertile environment amplifies his ability to realize the full potential of his research.

In all these respects, the Beneficiary is undoubtedly well-positioned. He combines academic brilliance, proven research output, practical technical skills, and innovation-oriented thinking. He is not merely a student of the field; he is a contributor advancing the state of the art, as evidenced by his publications and patent application.


\clearpage

{\bf \underline{Section 4.} Benefit To The United States Of Waiving The Job Offer And Labor Certification Requirements. }

Granting a National Interest Waiver (NIW) to the Beneficiary and removing the traditional job offer and labor certification constraints is not merely a procedural shortcut; it is a strategic decision that aligns directly with U.S. national interests. The Dhanasar framework and related legal precedents support waivers in scenarios where conventional labor market tests would impede progress in fields that are of exceptional importance, especially when time-sensitive, high-level expertise is at stake. In this case, the Beneficiary’s unique background in EEG-based epileptic seizure prediction and advanced deep learning models exemplifies such a scenario. By bypassing the conventional hurdles, the United States positions itself to reap immediate and long-term gains:

{\bf 4.1 Catalyzing Innovations in a Rapidly Evolving, Specialized Field.}

The Beneficiary operates at the cutting edge of a multidisciplinary niche—epilepsy seizure prediction leveraging sophisticated EEG signal processing and state-of-the-art Convolutional Neural Networks. This expertise is far from commonplace, even within the robust U.S. talent pool of data scientists, engineers, and medical researchers. Labor certification presupposes a relatively static skill set and standardized job definition, yet frontier research seldom fits neatly into existing occupational categories. By granting the NIW, the U.S. avoids delaying urgent and evolving research efforts. Instead of devoting months or years to verifying the domestic availability of an exact skill match—an inherently flawed approach given the field’s novelty—the nation can harness these capabilities immediately. This rapid integration ensures that groundbreaking work continues unfettered, accelerating the development and validation of advanced seizure prediction methods.

{\bf 4.2 Immediate Integration into High-Impact Research and Healthcare Systems.}

The U.S. healthcare and research ecosystems are not static; they are dynamic environments where timely expertise can be critical. NIH-funded labs might currently be testing various predictive algorithms, startups could be poised to incorporate next-generation EEG analytics into wearable technology, and major hospital networks may be initiating clinical trials that rely on state-of-the-art seizure prediction models. Each of these endeavors stands to benefit directly from the Beneficiary’s specialized skill set. Requiring a labor certification would introduce months-long delays, risking missed funding cycles, jeopardizing project timelines, and potentially slowing the translation of research findings into clinical practice. Waiving these requirements, on the other hand, allows the Beneficiary to join and strengthen these ventures swiftly, ensuring that projects maintain momentum, resources are used efficiently, and promising leads are not lost.

{\bf 4.3 Promoting Strategic Flexibility for Maximum National Impact.}

The American innovation ecosystem thrives when talented individuals can allocate their skills to projects offering the greatest societal benefit. Restrictive requirements tying a researcher to a single employer or a predefined position may limit their ability to respond to evolving research needs, shifting priorities, or emergent opportunities. By granting the NIW, the U.S. ensures that the Beneficiary’s contributions are not confined by bureaucratic constraints. This flexibility enables him to collaborate across sectors—be it academia, industry, government labs, or non-profit healthcare initiatives—focusing on endeavors that most directly advance patient care, reduce healthcare costs, and foster innovation. In other words, the NIW unleashes the full potential of a highly skilled individual to serve the national interest where and when it is most needed.


{\bf 4.4 Strengthening the United States’ Global Leadership in Medical AI.}

The race to advance medical AI and neuro-engineering is global in nature. Nations worldwide actively compete to attract and retain top-tier researchers capable of delivering transformative breakthroughs. By easing entry barriers for a candidate whose work is inherently aligned with U.S. national interests, the country cements its reputation as a premier destination for cutting-edge scientific talent. Such strategic talent acquisition ensures that the U.S. remains at the forefront of medical technology innovation, safeguarding its leadership position and bolstering its competitiveness in a rapidly expanding global market.


{\bf 4.5 Fostering Healthcare Cost Savings and Long-Term Economic Benefits.}

Effective seizure prediction can preempt costly emergency interventions and hospitalizations. By integrating advanced predictive algorithms into clinical workflows, it is possible to minimize the frequency and severity of acute epileptic events. Over time, this prevention-focused approach translates into significant cost savings for hospitals, insurers, patients, and public healthcare programs such as Medicare and Medicaid. Reducing the financial burden associated with epilepsy management has a domino effect: it improves patient outcomes, enhances quality of life, and ensures healthcare resources are allocated more efficiently. Lower healthcare expenses and improved patient care are prime examples of national interest objectives that a NIW is designed to support.


{\bf 4.6 Upholding Legislative Intent and the Dhanasar Framework.}

The National Interest Waiver category exists to streamline entry for individuals whose expertise transcends conventional labor market considerations. The Dhanasar decision and prior precedents clarify that labor certification may be unnecessary—and even counterproductive—when an applicant’s qualifications and contributions serve a compelling national interest. Here, the Beneficiary’s work in improving seizure prediction aligns precisely with this principle. By granting the NIW, USCIS upholds the spirit of these legislative and judicial guidelines, affirming that America welcomes individuals who can meaningfully advance public welfare and scientific progress.

{\bf 4.7 Creating Complementary Opportunities for U.S. Workers.}

Far from displacing American talent, the Beneficiary’s arrival would enhance and expand research teams, encourage interdisciplinary collaboration, and spark additional funding avenues. When an exceptionally skilled individual contributes to new breakthroughs, it often prompts hiring more research assistants, data analysts, clinical trial coordinators, and technical support staff. In this way, the Beneficiary’s expertise acts as a catalyst, driving growth and job creation rather than competing with U.S. workers. Ultimately, the improved healthcare solutions resulting from this research benefit not only patients but the economy as a whole, reinforcing the idea that the NIW aligns with broader national employment interests.

By waiving the job offer and labor certification requirements, the United States ensures that a highly specialized researcher can immediately contribute to fields with direct relevance to public health, national competitiveness, and economic well-being. This policy decision accelerates innovation, aligns with legislative intent, and ultimately enhances the nation’s leadership in cutting-edge medical science. In short, granting this NIW represents a strategic, future-oriented choice fully consistent with America’s best interests.

\clearpage

{\bf \underline{Section 5.} Concluding Remarks. }

“Clearly, Mr. Hu is an exceptionally talented mathematician and engineer, and he would be a great asset to the United States if he continued working here. If he enters academics as a Engineer, as planned, then he will make important contributions to the medical artificial intelligence for State and the US, training future generations of engineers among the Veterans and the native population. If he stays in the industry, he will create new commercial opportunities based on his advanced knowledge of next-generation AI technology and the safety of large computational systems. In short, Mr. Hu is already a leader in main areas of immense importance to our economy and national security, and his leadership position will inevitably increase. Please give favorable consideration to his Green Card application.” ({\it Exhibit 2, a letter from Professor A}) 

“I endorse the immigration petition by Mr. Hu and ask you to decide favorably on his behalf so that he can continue his important research without delays and distractions.” ({\it Exhibit 4, a letter from Professor B}) 

"Mr. Hu offers a unique skillset to the American scientific community. He is a creative engineer with an unusually keen attention to detail. I strongly support his application." ({\it Exhibit 3, a letter from F}) 

“Let me finish this letter with the statement that Mr. Hu is a brilliant young investigator in the fields of operations research and artificial intelligence. Granting him permanent residence in the U.S. will allow his work to proceed uninterrupted so he can concentrate on the application of his skills and knowledge to solving major computational problems.” ({\it Exhibit 5, a letter from Dr. C}) 

“I wholeheartedly endorse Mr. Hu's application for the EB-2 NIW petition for Permanent Residence. His advanced research and demonstrated leadership in his field make him a strong candidate, and I am confident that his ongoing contributions will substantially benefit our nation.” ({\it Exhibit 7, a letter from Professor E}) 

“In my opinion, Mr. Hu has made very significant discoveries in control and signal processing and helped in the advance of operations research. His outstanding abilities and expertise will be a huge asset to the science of computation in the USA. There is no doubt he will continue to have a major impact on electrical engineering, operations research, machine learning, and artificial intelligence. He will certainly contribute substantially to the well-being of American society and help sustain the USA as the leading light in world science.” ({\it Exhibit 6, a letter from Dr. D}) 

The Beneficiary’s research into EEG-based epileptic seizure prediction using advanced machine learning techniques, including stacked CNNs, directly addresses a critical national interest. By tackling a major healthcare challenge—epilepsy—and offering solutions that could improve patient independence, reduce healthcare spending, and enhance U.S. leadership in medical AI, his work holds substantial merit and national importance.

He has demonstrated, through exemplary academic performance, hands-on research experience, two peer-reviewed international publications, and a patent application, that he is eminently well-positioned to advance this field. He is not a novice but a researcher who has already achieved innovations recognized by the international scholarly community.

Requiring a job offer and labor certification would slow the pace of beneficial collaborations, limit his potential contributions, and ultimately serve as an unnecessary barrier to harnessing his specialized skill set for the U.S. public good. Granting a National Interest Waiver aligns perfectly with the legislative intent behind the NIW category and ensures that his talents can be immediately and fully leveraged to advance public health and scientific innovation in the United States.

For these reasons, I respectfully request that USCIS approve this petition and grant the National Interest Waiver for the Beneficiary. Should there be any need for additional information or clarification, I will be pleased to provide further documentation.
Thank you for your time and careful consideration.

Respectfully submitted,

\vspace{5\baselineskip}

Wenbin Hu\\
999 99th st,\\
Default City, FL, 99999\\
Tel. (999) 999-9999




\clearpage

{\bf Statement from Mr. Hu detailing plans on how he intends to continue work in the United States}

\label{plans}
December 31, 2024

My name is Wenbin Hu. I am writing this letter to provide a detailed overview of my plans on how I intend to continue my research and related endeavors in the United States, should my National Interest Waiver (NIW) petition be approved. As outlined in my I-140 petition, my area of focus is improving the prediction and understanding of epileptic seizures through advanced EEG-based machine learning techniques. By bringing my expertise to the U.S., I aim to drive meaningful progress in epilepsy care, contribute to medical AI innovation, and engage with various American institutions and enterprises dedicated to this field.

Below, I present my strategic approach and anticipated activities, collaborations, and contributions, reflecting how I envision establishing myself as a productive and influential researcher in the United States.


{\bf 1. Academic and Clinical Collaborations }

{\bf University Partnerships: }

I plan to affiliate with leading academic medical centers and universities known for epilepsy and neuroscience research, such as Johns Hopkins University, the Mayo Clinic, and the Cleveland Clinic. Working within their neurology and biomedical engineering departments will provide access to large EEG datasets and ongoing clinical trials. This environment will allow me to validate my algorithms against diverse patient populations, refine predictive accuracy, and rapidly incorporate clinical feedback.

{\bf NIH and NINDS Initiatives: }

I aim to align my work with NIH-funded projects, including initiatives by the National Institute of Neurological Disorders and Stroke (NINDS), and possibly the BRAIN Initiative. By joining consortia focused on improving diagnostic and therapeutic approaches for epilepsy, I can integrate my models into federally supported research, ensuring my contributions directly advance U.S. public health goals.

{\bf 2. Industry and Startup Engagement }

{\bf Medical Device Firms and AI Companies: }

The U.S. has a thriving healthcare technology sector. I will seek collaborations with startups and established med-tech companies developing EEG-based monitoring devices or AI-driven healthcare platforms. Integrating my seizure prediction algorithms into their product pipelines will help accelerate commercialization. This approach ensures that the benefits of my research—improved accuracy, timely warnings, and reduced false alarms—are realized swiftly and made available to patients across the country.

{\bf Technology Transfer and Entrepreneurship: } 

If appropriate, I may pursue technology licensing agreements or partner with university technology transfer offices. These pathways could lead to SBIR grants, private investment, or co-founding a startup focused on delivering reliable seizure prediction tools to U.S. clinics, hospitals, and telemedicine services.

{\bf 3. Funding and Grant Strategies }

{\bf Federal Grants (NIH, NSF): }

I will apply for NIH and NSF grants that support innovative diagnostic tools, AI in healthcare, and neurological research. Successful grant proposals will help me scale up my work, fund additional data analyses, and perform long-term clinical validations.

{\bf Foundation and Non-Profit Support: }

I will also seek grants from organizations like the Epilepsy Foundation, which often provide seed funding for innovative approaches. Such awards can support pilot studies and generate preliminary data critical for securing larger federal grants.

{\bf 4. Clinical Implementation and Validation }

{\bf Pilot Studies in U.S. Hospitals: }

My immediate goal is to test my EEG-based models in real clinical environments. Partnering with epilepsy monitoring units will allow me to measure algorithmic performance on patients undergoing routine evaluation, assessing lead times, false positives, and ease of integration into clinical workflows.

{\bf Regulatory Pathways and FDA Compliance: }

I intend to follow FDA guidelines for AI-based medical tools, ensuring that my algorithms meet safety, transparency, and efficacy standards. Early engagement with FDA frameworks will streamline future device approvals, facilitating quicker adoption of prediction tools in everyday patient care.

{\bf 5. Education, Training, and Dissemination }

{\bf Mentorship and Teaching: }

By collaborating with U.S. universities, I can mentor graduate students and research assistants, helping develop a skilled talent pool versed in medical AI and EEG analysis. This capacity-building aligns with U.S. interests in sustaining a robust research community.

{\bf Conferences and Workshops: }

I will present my work at national conferences on biomedical engineering, neuroscience, and AI in healthcare. Sharing methodologies, open-source code, and research findings will foster community engagement, stimulate feedback, and encourage the formation of new collaborations.

{\bf 6. Long-Term Vision and Broader Impact } 

{\bf Extending to Other Neurological Conditions: }

While epilepsy prediction is my initial focus, the underlying deep learning methods I use can be adapted to detect early markers of other neurological disorders. Over time, I plan to expand my scope to conditions like Parkinson’s disease or Alzheimer’s, contributing to a broader range of U.S. health priorities.

{\bf Policy and Standards Contribution: }

As my work matures, I may engage with patient advocacy groups, healthcare policy makers, and regulatory bodies. By advising on best practices and ethics in AI-driven neurology, I can help shape national standards, ensuring that these emerging technologies serve patient interests and align with U.S. healthcare policies.

My strategy for continuing work in the United States is structured, impact-driven, and aligned with national priorities. By establishing partnerships with leading research institutions, contributing to federally funded programs, collaborating with industry, securing diverse funding, achieving clinical validation, and participating in educational and policy discussions, I will ensure that my expertise in EEG-based seizure prediction delivers tangible benefits. This integrated approach will improve patient outcomes, reduce healthcare burdens, enhance American leadership in medical innovation, and advance the nation’s public health objectives.

Thank you for considering this plan. I remain at your disposal for any further details or clarifications.

Respectfully,

[Name of Beneficiary]

\vspace{5\baselineskip}

Wenbin Hu\\
999 99th st,\\
Default City, FL, 99999\\
Tel. (999) 999-9999


\clearpage

{\bf List of Exhibits}
\label{exhib}

\begin{enumerate}[label={Exhibit \arabic*:}]
    \item Curriculum Vitae of Mr. Wenbin Hu
    \item Letter of Professor Jiuwen Cao, University of Hangzhou Dianzi University
    \item Letter of Ms. Yibing Sheng, CEO of Hangzhou Yanzhi Technology Co., Ltd
    \item Letter of Mr. Xiangshao Liu, Principal Engineer of Huawei Technologies Co., Ltd
    \item Letter of Professor D, University of XXX
    \item Letter of Professor E, University of E 
    \item Letter of Dr. C, University of XXX
    \item Bachelor degree of Mr. Hu 
    \item Master degree of Mr. Hu 
    \item Academic Ranking of World Universities 2024
    \item Statistical Bulletin on the Development of National Education in China
    \item Graduate transcript
    \item Bank Revenue Flow 2019
    \item Subsidiary revenue proof
    \item Proof of Parent company's Income in 2024
    \item Certificate of dividend income after tax
    \item Average survey for five years experiences employee in 2024
    \item Citation statistical results
    \item Citation reports
    \item Review assignments completed by Mr. Hu and review invatation by IJCNN and AIMed
    \item Journal Ranking
    \item A letter from Editor of the AI MED journal
    \item First pages of 3 papers co-authored by Mr. Hu
    \item Top venues rankings
    \item Employee Certificate of Incumbency
    \item 12 Leading Chinese Companies in the AI Large Model Sector
    \item Epilepsy Facts and Stats by CDC
    \item H.R.10210 - National Plan for Epilepsy Act
    \item Focus On Epilepsy Research-NINDS
    \item Tools and Technologies for Brain Cells and Circuits-BRAIN
    \item Artificial Intelligence (AI) Strategy-HHS
    \item National Artificial Intelligence Research and Development Strategic Plan 2023
    \item From smart shoes to EEG-13 companies pioneering wearable medical devices-ESCATEC
    \item Epilepsy in the Workplace and the ADA
    \item The Epilepsy Leadership Council (ELC) Policy Agenda
    \item Chinese patent authored by Mr. Hu
\end{enumerate}

\clearpage

\vspace*{\fill}
\begin{center}

{\LARGE \bf
Exhibit 1
}

\vspace{10\baselineskip}

{\large Curriculum Vitae of Mr. Hu}

\end{center}
\vspace*{\fill}

% \includepdf[pages=-,pagecommand={},width=1.3\textwidth]{Exhibits/CV.pdf}

\vspace*{\fill}
\begin{center}

{\LARGE \bf
Exhibit 2
}

\vspace{10\baselineskip}

{\large Letter of Professor Jiuwen Cao, University of Hangzhou Dianzi University}

\end{center}
\vspace*{\fill}
f
% \includepdf[pages=-,pagecommand={},width=1.3\textwidth]{Exhibits/recomendations/Cao.pdf}

\vspace*{\fill}
\begin{center}

{\LARGE \bf
Exhibit 3
}

\vspace{10\baselineskip}

{\large Letter of Ms. Yibing Sheng, CEO of Hangzhou Yanzhi Technology Co., Ltd.}

\end{center}
\vspace*{\fill}

 % \includepdf[pages=-,pagecommand={},width=1.3\textwidth]{Exhibits/recomendations/Sheng.pdf}


\vspace*{\fill}
\begin{center}

{\LARGE \bf
Exhibit 4
}

\vspace{10\baselineskip}

{\large Letter of Mr. Xiangshao Liu, Principal Engineer of Huawei Technologies Co., Ltd}

\end{center}
\vspace*{\fill}


% \includepdf[pages=-,pagecommand={},width=1.3\textwidth]{Exhibits/recomendations/Liu.pdf}



\vspace*{\fill}
\begin{center}

{\LARGE \bf
Exhibit 5
}

\vspace{10\baselineskip}

{\large  Letter of Professor D, University of XXX}

\end{center}
\vspace*{\fill}


% \includepdf[pages=-,pagecommand={},width=1.2\textwidth]{Exhibits/recomendations/D.pdf}




\vspace*{\fill}
\begin{center}

{\LARGE \bf
Exhibit 6
}

\vspace{10\baselineskip}

{\large Letter of Professor E, University of E}

\end{center}
\vspace*{\fill}

% \includepdf[pages=-,pagecommand={},width=1.3\textwidth]{Exhibits/recomendations/E.pdf}




\vspace*{\fill}
\begin{center}

{\LARGE \bf
Exhibit 7
}

\vspace{10\baselineskip}

{\large Letter of Dr. C, University of XXX}

\end{center}
\vspace*{\fill}

% \includepdf[pages=-,pagecommand={},width=1.3\textwidth]{Exhibits/recomendations/C.pdf}




\vspace*{\fill}
\begin{center}

{\LARGE \bf
Exhibit 8
}

\vspace{10\baselineskip}

{\large Bachelor degree of Mr. Hu}

\end{center}
\vspace*{\fill}


% \includepdf[pages=-,pagecommand={},width=1.3\textwidth]{Exhibits/Bachelor degree.pdf}


\vspace*{\fill}
\begin{center}

{\LARGE \bf
Exhibit 9
}

\vspace{10\baselineskip}

{\large Master degree of Mr. Hu .}

\end{center}
\vspace*{\fill}

% \includepdf[pages=-,pagecommand={},width=1.3\textwidth]{Exhibits/Master degree.pdf}




\vspace*{\fill}
\begin{center}

{\LARGE \bf
Exhibit 10
}

\vspace{10\baselineskip}

{\large Academic Ranking of World Universities 2024}

\end{center}
\vspace*{\fill}

 \includepdf[pages=-,pagecommand={},width=1.3\textwidth]{Exhibits/Academic Ranking of World Universities 2024.pdf}

\vspace*{\fill}
\begin{center}

{\LARGE \bf
Exhibit 11
}

\vspace{10\baselineskip}

{\large Statistical Bulletin on the Development of National Education in China}

\end{center}
\vspace*{\fill}

 \includepdf[pages=-,pagecommand={},width=1.3\textwidth]{Exhibits/Statistical Bulletin on the Development of National Education in China.pdf}

\vspace*{\fill}
\begin{center}

{\LARGE \bf
Exhibit 12
}

\vspace{10\baselineskip}

{\large Graduate transcript}

\end{center}
\vspace*{\fill}


 \includepdf[pages=-,pagecommand={},width=1.3\textwidth]{Exhibits/Graduate transcript.pdf}



\vspace*{\fill}
\begin{center}

{\LARGE \bf
Exhibit 13
}

\vspace{10\baselineskip}

{\large  Bank Revenue Flow 2019}

\end{center}
\vspace*{\fill}


% \includepdf[pages=-,pagecommand={},width=1.3\textwidth]{Exhibits/Bank Revenue Flow 2019.pdf}

% \includepdf[pages=-,pagecommand={},width=1.3\textwidth]{Exhibits/my papers/2.pdf}




\vspace*{\fill}
\begin{center}

{\LARGE \bf
Exhibit 14
}

\vspace{10\baselineskip}

{\large Subsidiary revenue proof}

\end{center}
\vspace*{\fill}

% \includepdf[pages=-,pagecommand={},width=1.3\textwidth]{Exhibits/Subsidiary revenue proof.pdf}


\vspace*{\fill}
\begin{center}

{\LARGE \bf
Exhibit 15
}

\vspace{10\baselineskip}

{\large Proof of Parent company's Income in 2024}

\end{center}
\vspace*{\fill}

% \includepdf[pages=-,pagecommand={},width=1.3\textwidth]{Exhibits/Proof of Parent company's Income in 2024.pdf}


\vspace*{\fill}
\begin{center}

{\LARGE \bf
Exhibit 16
}

\vspace{10\baselineskip}

{\large  Certificate of dividend income after tax}

\end{center}
\vspace*{\fill}

% \includepdf[pages=-,pagecommand={},width=1.3\textwidth]{Exhibits/Certificate of dividend income after tax.pdf}

% \includepdf[pages=-,pagecommand={},width=1.3\textwidth]{Exhibits/review invites/2.pdf}


\vspace*{\fill}
\begin{center}

{\LARGE \bf
Exhibit 17
}

\vspace{10\baselineskip}

{\large Average survey for five years experiences employee in 2024}

\end{center}
\vspace*{\fill}

% \includepdf[pages=-,pagecommand={},width=1.3\textwidth]{Exhibits/Average survey for five years experiences employee in 2024.pdf}

\vspace*{\fill}
\begin{center}

{\LARGE \bf
Exhibit 18
}

\vspace{10\baselineskip}

{\large Citation statistical results}

\end{center}
\vspace*{\fill}

\includepdf[pages=-,pagecommand={},width=1.3\textwidth]{Exhibits/Citation statistical results.pdf}


\vspace*{\fill}
\begin{center}

{\LARGE \bf
Exhibit 19
}

\vspace{10\baselineskip}

{\large Citation reports}

\end{center}
\vspace*{\fill}

\includepdf[pages={1,3,11},pagecommand={},width=1.3\textwidth]{Citation Papers/Notable_citation_papers/1Landscape Perception Identification and Classification Based on Electroencephalogram (EEG) Features.pdf}
% original page: 1,3,11; citation page:2 , section 2.4
\includepdf[pages={1,6, 17},pagecommand={},width=1.3\textwidth]{Citation Papers/Notable_citation_papers/2Epilepsy EEG Seizure Prediction Based on the Combination of Graph Convolutional Neural Network Combined with Long- and Short-Term Memory Cell Network.pdf}
% original page: 1,6, 17, citation page: 2,  section 3.2.3
\includepdf[pages={1,12,13},pagecommand={},width=1.3\textwidth]{Citation Papers/Notable_citation_papers/3Hierarchy Graph Convolution Network and Tree Classification for Epileptic Detection on Electroencephalography Signals.pdf}
% original page:1,12,13,  citation page: 2, section D
\includepdf[pages={1,2,5},pagecommand={},width=1.3\textwidth]{Citation Papers/Notable_citation_papers/4Epileptic Seizure Detection and Anticipation using Deep Learning with Ordered Encoding of Spectrogram Features.pdf}
%original: 1,2,5, citation page: 2, section II
\includepdf[pages={1,3, 8},pagecommand={},width=1.3\textwidth]{Citation Papers/Notable_citation_papers/5EpilepsyNet-Novel automated detection of epilepsy using transformer model with EEG signals from 121 patient population.pdf}
% original: 1,3, 8  citation: 2, Table 1


\includepdf[pages={1,3,19,},pagecommand={},width=1.3\textwidth]{Citation Papers/Notable_citation_papers/6A Mutual Information-Based Many-Objective Optimization Method for EEG Channel Selection in the Epileptic Seizure Prediction Task.pdf}
%original:1,3,19, citation:2,  Patient-Independents Seizure Prediction with All channels
\includepdf[pages={1,2,12},pagecommand={},width=1.3\textwidth]{Citation Papers/Notable_citation_papers/7Geometric Deep Learning for Subject Independent Epileptic Seizure Prediction Using Scalp EEG Signals.pdf}
%original:1,2,12   citation:2, II 
\includepdf[pages={1,2,22},pagecommand={},width=1.3\textwidth]{Citation Papers/Notable_citation_papers/8An efficient epileptic seizure classification system using empirical wavelet transform and multi-fuse reduced deep convolutional neural network with digital implementation.pdf}
%oridinal:1,2,22, citation:2
\includepdf[pages={1,2,26},pagecommand={},width=1.3\textwidth]{Citation Papers/Notable_citation_papers/9Patient-specific solution of the electrocorticography forward problem in deforming brain.pdf}
%original:1,2,26, citation:2, 
\includepdf[pages={1,15,17},pagecommand={},width=1.3\textwidth]{Citation Papers/Notable_citation_papers/10A review of epilepsy detection and prediction methods based on EEG signal processing and deep learning.pdf}
%original:1,15,17, citation:2


\includepdf[pages={1,3,7},pagecommand={},width=1.3\textwidth]{Citation Papers/Notable_citation_papers/11Analysis Between ELM and ANN in EMG Signals Obtained for the Control of a Robotic Hand Prosthesis.pdf}
%1,3,7,  2, section 2.1
\includepdf[pages={1,3,14},pagecommand={},width=1.3\textwidth]{Citation Papers/Notable_citation_papers/12An Intelligent Epileptic Prediction System Based on Synchrosqueezed Wavelet Transform and Multi-Level Feature CNN for Smart Healthcare IoT.pdf}
%1,3,14, citation:2
\includepdf[pages={1,2,3,11},pagecommand={},width=1.3\textwidth]{Citation Papers/Notable_citation_papers/13Predicting epileptic seizures based on EEG signals using spatial depth features of a 3D-2D hybrid CNN.pdf}
%1,2,3,11, citation:2,3
\includepdf[pages={1,2,12},pagecommand={},width=1.3\textwidth]{Citation Papers/Notable_citation_papers/14Applications of Artificial Intelligence in Automatic Detection of Epileptic Seizures Using EEG Signals- A Review.pdf}
%1,2,12, citation:2
\includepdf[pages={1,2,17},pagecommand={},width=1.3\textwidth]{Citation Papers/Notable_citation_papers/15Automatic epileptic seizure detection based on persistent homology.pdf}
%1,2,17, citation:2


\vspace*{\fill}
\begin{center}
{\LARGE \bf
Exhibit 20
}

\vspace{10\baselineskip}

{\large Review assignments completed by Mr. Hu and review invatation by IJCNN and AIMed}

\end{center}
\vspace*{\fill}

% \includepdf[pages=-,pagecommand={},width=1.3\textwidth]{Exhibits/Review assignments completed by Mr. Hu and review invatation by IJCNN and AIMed.pdf}

\vspace*{\fill}
\begin{center}
{\LARGE \bf
Exhibit 21
}

\vspace{10\baselineskip}

{\large Journal Ranking}

\end{center}
\vspace*{\fill}

%\includepdf[pages=-,pagecommand={},width=1.3\textwidth]{Exhibits/Journal Ranking.pdf}


\vspace*{\fill}
\begin{center}
{\LARGE \bf
Exhibit 22
}

\vspace{10\baselineskip}

{\large A letter from Editor of the AI MED jorunal}

\end{center}
\vspace*{\fill}


% \includepdf[pages=-,pagecommand={},width=1.3\textwidth]{Exhibits/A letter from Editor of the AI MED jorunal.pdf}



\vspace*{\fill}
\begin{center}
{\LARGE \bf
Exhibit 23
}

\vspace{10\baselineskip}

{\large First pages of 3 papers co-authored by Mr. Hu}

\end{center}
\vspace*{\fill}

  \includepdf[pages=1,pagecommand={},width=1.3\textwidth]{Exhibits/publications/Epileptic Signal Classification with Deep EEG Features by Stacked CNNs.pdf}
  
 \includepdf[pages=1,pagecommand={},width=1.3\textwidth]{Exhibits/publications/Mean amplitude spectrum based epileptic state classification for seizure prediction using convolutional neural networks.pdf}
 



\vspace*{\fill}
\begin{center}
{\LARGE \bf
Exhibit 24
}

\vspace{10\baselineskip}

{\large Top venues rankings}

\end{center}
\vspace*{\fill}


% \includepdf[pages=-,pagecommand={},width=1.3\textwidth]{Exhibits/Top venues rankings.pdf}

% \includepdf[pages=-,pagecommand={},width=1.3\textwidth]{Exhibits/Top Venues/Journal Rankings on Artificial Intelligence.pdf}

% \includepdf[pages=-,pagecommand={},width=1.3\textwidth]{Exhibits/Top Venues/Mathematical Optimization - Google Scholar Metrics.pdf}

% \includepdf[pages=-,pagecommand={},width=1.3\textwidth]{Exhibits/Top Venues/Power Engineering - Google Scholar Metrics.pdf}



\vspace*{\fill}
\begin{center}
{\LARGE \bf
Exhibit 25
}

\vspace{10\baselineskip}

{\large Employee Certificate of Incumbency}

\end{center}
\vspace*{\fill}

%\includepdf[pages=-,pagecommand={},width=1.3\textwidth]{Exhibits/Employee Certificate of Incumbency.pdf}




\vspace*{\fill}
\begin{center}
{\LARGE \bf
Exhibit 26
}

\vspace{10\baselineskip}

{\large 12 Leading Chinese Companies in the AI Large Model Sector}

\end{center}
\vspace*{\fill}

\includepdf[pages={1,2,3,17,18},pagecommand={},width=1.2\textwidth]{Exhibits/12 Leading Chinese Companies in the AI Large Model Sector.pdf}




\vspace*{\fill}
\begin{center}
{\LARGE \bf
Exhibit 27
}

\vspace{10\baselineskip}

{\large Epilepsy Facts and Stats by CDC }

\end{center}
\vspace*{\fill}

\includepdf[pages=-,pagecommand={},width=1.3\textwidth]{Exhibits/Epilepsy Facts and Stats-According to CDC.pdf}




\vspace*{\fill}
\begin{center}

{\LARGE \bf
Exhibit 28
}

\vspace{10\baselineskip}

{\large H.R.10210 - National Plan for Epilepsy Act}

\end{center}
\vspace*{\fill}

\includepdf[pages=-,pagecommand={},width=1.3\textwidth]{Exhibits/H.R.10210 - National Plan for Epilepsy Act.pdf}




\vspace*{\fill}
\begin{center}

{\LARGE \bf
Exhibit 29
}

\vspace{10\baselineskip}

{\large Focus On Epilepsy Research-NINDS}

\end{center}
\vspace*{\fill}

\includepdf[pages={1,2,3},pagecommand={},width=1.3\textwidth]{Exhibits/Focus On Epilepsy Research-NINDS.pdf}




\vspace*{\fill}
\begin{center}

{\LARGE \bf
Exhibit 30
}

\vspace{10\baselineskip}

{\large Tools and Technologies for Brain Cells and Circuits-BRAIN}

\end{center}
\vspace*{\fill}

 \includepdf[pages=-,pagecommand={},width=1.2\textwidth]{Exhibits/Tools and Technologies for Brain Cells and Circuits-BRAIN.pdf}

\vspace*{\fill}
\begin{center}

{\LARGE \bf
Exhibit 31
}

\vspace{10\baselineskip}

{\large Artificial Intelligence (AI) Strategy-HHS}

\end{center}
\vspace*{\fill}

 \includepdf[pages=-,pagecommand={},width=1.2\textwidth]{Exhibits/Artificial Intelligence (AI) Strategy-HHS.pdf}

\vspace*{\fill}
\begin{center}

{\LARGE \bf
Exhibit 32
}

\vspace{10\baselineskip}

{\large National Artificial Intelligence Research and Development Strategic Plan 2023}

\end{center}
\vspace*{\fill}

%\includepdf[pages=-,pagecommand={},width=1.2\textwidth]{Exhibits/National Artificial Intelligence Research and Development Strategic Plan 2023.pdf}

\vspace*{\fill}
\begin{center}

{\LARGE \bf
Exhibit 33
}

\vspace{10\baselineskip}

{\large From smart shoes to EEG-13 companies pioneering wearable medical devices-ESCATEC}

\end{center}
\vspace*{\fill}

 \includepdf[pages=-,pagecommand={},width=1.2\textwidth]{Exhibits/From smart shoes to EEG-13 companies pioneering wearable medical devices-ESCATEC.pdf}


\vspace*{\fill}
\begin{center}

{\LARGE \bf
Exhibit 34
}

\vspace{10\baselineskip}

{\large Epilepsy in the Workplace and the ADA}

\end{center}
\vspace*{\fill}

 %\includepdf[pages=-,pagecommand={},width=1.2\textwidth]{Exhibits/Epilepsy in the Workplace and the ADA.pdf}

\vspace*{\fill}
\begin{center}

{\LARGE \bf
Exhibit 35
}

\vspace{10\baselineskip}

{\large The Epilepsy Leadership Council (ELC) Policy Agenda}

\end{center}
\vspace*{\fill}

 \includepdf[pages=-,pagecommand={},width=1.2\textwidth]{Exhibits/The Epilepsy Leadership Council (ELC) Policy Agenda.pdf}

\vspace*{\fill}
\begin{center}

{\LARGE \bf
Exhibit 36
}

\vspace{10\baselineskip}

{\large Chinese patent authored by Mr. Hu}

\end{center}
\vspace*{\fill}

% \includepdf[pages=-,pagecommand={},width=1.2\textwidth]{Exhibits/Chinese patent authored by Mr. Hu.pdf}

\end{document}
